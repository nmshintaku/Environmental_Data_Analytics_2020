\documentclass[]{article}
\usepackage{lmodern}
\usepackage{amssymb,amsmath}
\usepackage{ifxetex,ifluatex}
\usepackage{fixltx2e} % provides \textsubscript
\ifnum 0\ifxetex 1\fi\ifluatex 1\fi=0 % if pdftex
  \usepackage[T1]{fontenc}
  \usepackage[utf8]{inputenc}
\else % if luatex or xelatex
  \ifxetex
    \usepackage{mathspec}
  \else
    \usepackage{fontspec}
  \fi
  \defaultfontfeatures{Ligatures=TeX,Scale=MatchLowercase}
\fi
% use upquote if available, for straight quotes in verbatim environments
\IfFileExists{upquote.sty}{\usepackage{upquote}}{}
% use microtype if available
\IfFileExists{microtype.sty}{%
\usepackage{microtype}
\UseMicrotypeSet[protrusion]{basicmath} % disable protrusion for tt fonts
}{}
\usepackage[margin=2.54cm]{geometry}
\usepackage{hyperref}
\hypersetup{unicode=true,
            pdftitle={Assignment 6: GLMs week 1 (t-test and ANOVA)},
            pdfauthor={Nikki Shintaku},
            pdfborder={0 0 0},
            breaklinks=true}
\urlstyle{same}  % don't use monospace font for urls
\usepackage{color}
\usepackage{fancyvrb}
\newcommand{\VerbBar}{|}
\newcommand{\VERB}{\Verb[commandchars=\\\{\}]}
\DefineVerbatimEnvironment{Highlighting}{Verbatim}{commandchars=\\\{\}}
% Add ',fontsize=\small' for more characters per line
\usepackage{framed}
\definecolor{shadecolor}{RGB}{248,248,248}
\newenvironment{Shaded}{\begin{snugshade}}{\end{snugshade}}
\newcommand{\AlertTok}[1]{\textcolor[rgb]{0.94,0.16,0.16}{#1}}
\newcommand{\AnnotationTok}[1]{\textcolor[rgb]{0.56,0.35,0.01}{\textbf{\textit{#1}}}}
\newcommand{\AttributeTok}[1]{\textcolor[rgb]{0.77,0.63,0.00}{#1}}
\newcommand{\BaseNTok}[1]{\textcolor[rgb]{0.00,0.00,0.81}{#1}}
\newcommand{\BuiltInTok}[1]{#1}
\newcommand{\CharTok}[1]{\textcolor[rgb]{0.31,0.60,0.02}{#1}}
\newcommand{\CommentTok}[1]{\textcolor[rgb]{0.56,0.35,0.01}{\textit{#1}}}
\newcommand{\CommentVarTok}[1]{\textcolor[rgb]{0.56,0.35,0.01}{\textbf{\textit{#1}}}}
\newcommand{\ConstantTok}[1]{\textcolor[rgb]{0.00,0.00,0.00}{#1}}
\newcommand{\ControlFlowTok}[1]{\textcolor[rgb]{0.13,0.29,0.53}{\textbf{#1}}}
\newcommand{\DataTypeTok}[1]{\textcolor[rgb]{0.13,0.29,0.53}{#1}}
\newcommand{\DecValTok}[1]{\textcolor[rgb]{0.00,0.00,0.81}{#1}}
\newcommand{\DocumentationTok}[1]{\textcolor[rgb]{0.56,0.35,0.01}{\textbf{\textit{#1}}}}
\newcommand{\ErrorTok}[1]{\textcolor[rgb]{0.64,0.00,0.00}{\textbf{#1}}}
\newcommand{\ExtensionTok}[1]{#1}
\newcommand{\FloatTok}[1]{\textcolor[rgb]{0.00,0.00,0.81}{#1}}
\newcommand{\FunctionTok}[1]{\textcolor[rgb]{0.00,0.00,0.00}{#1}}
\newcommand{\ImportTok}[1]{#1}
\newcommand{\InformationTok}[1]{\textcolor[rgb]{0.56,0.35,0.01}{\textbf{\textit{#1}}}}
\newcommand{\KeywordTok}[1]{\textcolor[rgb]{0.13,0.29,0.53}{\textbf{#1}}}
\newcommand{\NormalTok}[1]{#1}
\newcommand{\OperatorTok}[1]{\textcolor[rgb]{0.81,0.36,0.00}{\textbf{#1}}}
\newcommand{\OtherTok}[1]{\textcolor[rgb]{0.56,0.35,0.01}{#1}}
\newcommand{\PreprocessorTok}[1]{\textcolor[rgb]{0.56,0.35,0.01}{\textit{#1}}}
\newcommand{\RegionMarkerTok}[1]{#1}
\newcommand{\SpecialCharTok}[1]{\textcolor[rgb]{0.00,0.00,0.00}{#1}}
\newcommand{\SpecialStringTok}[1]{\textcolor[rgb]{0.31,0.60,0.02}{#1}}
\newcommand{\StringTok}[1]{\textcolor[rgb]{0.31,0.60,0.02}{#1}}
\newcommand{\VariableTok}[1]{\textcolor[rgb]{0.00,0.00,0.00}{#1}}
\newcommand{\VerbatimStringTok}[1]{\textcolor[rgb]{0.31,0.60,0.02}{#1}}
\newcommand{\WarningTok}[1]{\textcolor[rgb]{0.56,0.35,0.01}{\textbf{\textit{#1}}}}
\usepackage{graphicx,grffile}
\makeatletter
\def\maxwidth{\ifdim\Gin@nat@width>\linewidth\linewidth\else\Gin@nat@width\fi}
\def\maxheight{\ifdim\Gin@nat@height>\textheight\textheight\else\Gin@nat@height\fi}
\makeatother
% Scale images if necessary, so that they will not overflow the page
% margins by default, and it is still possible to overwrite the defaults
% using explicit options in \includegraphics[width, height, ...]{}
\setkeys{Gin}{width=\maxwidth,height=\maxheight,keepaspectratio}
\IfFileExists{parskip.sty}{%
\usepackage{parskip}
}{% else
\setlength{\parindent}{0pt}
\setlength{\parskip}{6pt plus 2pt minus 1pt}
}
\setlength{\emergencystretch}{3em}  % prevent overfull lines
\providecommand{\tightlist}{%
  \setlength{\itemsep}{0pt}\setlength{\parskip}{0pt}}
\setcounter{secnumdepth}{0}
% Redefines (sub)paragraphs to behave more like sections
\ifx\paragraph\undefined\else
\let\oldparagraph\paragraph
\renewcommand{\paragraph}[1]{\oldparagraph{#1}\mbox{}}
\fi
\ifx\subparagraph\undefined\else
\let\oldsubparagraph\subparagraph
\renewcommand{\subparagraph}[1]{\oldsubparagraph{#1}\mbox{}}
\fi

%%% Use protect on footnotes to avoid problems with footnotes in titles
\let\rmarkdownfootnote\footnote%
\def\footnote{\protect\rmarkdownfootnote}

%%% Change title format to be more compact
\usepackage{titling}

% Create subtitle command for use in maketitle
\providecommand{\subtitle}[1]{
  \posttitle{
    \begin{center}\large#1\end{center}
    }
}

\setlength{\droptitle}{-2em}

  \title{Assignment 6: GLMs week 1 (t-test and ANOVA)}
    \pretitle{\vspace{\droptitle}\centering\huge}
  \posttitle{\par}
    \author{Nikki Shintaku}
    \preauthor{\centering\large\emph}
  \postauthor{\par}
    \date{}
    \predate{}\postdate{}
  

\begin{document}
\maketitle

\hypertarget{overview}{%
\subsection{OVERVIEW}\label{overview}}

This exercise accompanies the lessons in Environmental Data Analytics on
t-tests and ANOVAs.

\hypertarget{directions}{%
\subsection{Directions}\label{directions}}

\begin{enumerate}
\def\labelenumi{\arabic{enumi}.}
\tightlist
\item
  Change ``Student Name'' on line 3 (above) with your name.
\item
  Work through the steps, \textbf{creating code and output} that fulfill
  each instruction.
\item
  Be sure to \textbf{answer the questions} in this assignment document.
\item
  When you have completed the assignment, \textbf{Knit} the text and
  code into a single PDF file.
\item
  After Knitting, submit the completed exercise (PDF file) to the
  dropbox in Sakai. Add your last name into the file name (e.g.,
  ``Salk\_A06\_GLMs\_Week1.Rmd'') prior to submission.
\end{enumerate}

The completed exercise is due on Tuesday, February 18 at 1:00 pm.

\hypertarget{set-up-your-session}{%
\subsection{Set up your session}\label{set-up-your-session}}

\begin{enumerate}
\def\labelenumi{\arabic{enumi}.}
\item
  Check your working directory, load the \texttt{tidyverse},
  \texttt{cowplot}, and \texttt{agricolae} packages, and import the
  NTL-LTER\_Lake\_Nutrients\_PeterPaul\_Processed.csv dataset.
\item
  Change the date column to a date format. Call up \texttt{head} of this
  column to verify.
\end{enumerate}

\begin{Shaded}
\begin{Highlighting}[]
\CommentTok{#1}
\KeywordTok{getwd}\NormalTok{()}
\end{Highlighting}
\end{Shaded}

\begin{verbatim}
## [1] "/Users/nikkishintaku/Desktop/Environmental872/Environmental_Data_Analytics_2020"
\end{verbatim}

\begin{Shaded}
\begin{Highlighting}[]
\KeywordTok{library}\NormalTok{(tidyverse)}
\KeywordTok{library}\NormalTok{(cowplot)}
\KeywordTok{library}\NormalTok{(agricolae)}
\KeywordTok{library}\NormalTok{(ggthemes)}

\NormalTok{lake_nutrients <-}\StringTok{ }\KeywordTok{read.csv}\NormalTok{(}\StringTok{"./Data/Processed/NTL-LTER_Lake_Nutrients_PeterPaul_Processed.csv"}\NormalTok{)}

\CommentTok{#2}
\NormalTok{lake_nutrients}\OperatorTok{$}\NormalTok{sampledate <-}\StringTok{ }\KeywordTok{as.Date}\NormalTok{(lake_nutrients}\OperatorTok{$}\NormalTok{sampledate , }\DataTypeTok{format =} \StringTok{"%Y-%m-%d"}\NormalTok{)}
\KeywordTok{head}\NormalTok{(lake_nutrients}\OperatorTok{$}\NormalTok{sampledate)}
\end{Highlighting}
\end{Shaded}

\begin{verbatim}
## [1] "1991-05-20" "1991-05-20" "1991-05-20" "1991-05-20" "1991-05-20"
## [6] "1991-05-20"
\end{verbatim}

\hypertarget{wrangle-your-data}{%
\subsection{Wrangle your data}\label{wrangle-your-data}}

\begin{enumerate}
\def\labelenumi{\arabic{enumi}.}
\setcounter{enumi}{2}
\tightlist
\item
  Wrangle your dataset so that it contains only surface depths and only
  the years 1993-1996, inclusive. Set month as a factor.
\end{enumerate}

\begin{Shaded}
\begin{Highlighting}[]
\CommentTok{#Wrangle the data}
\NormalTok{lake_nutrients_depth <-}\StringTok{ }
\StringTok{  }\NormalTok{lake_nutrients }\OperatorTok
\StringTok{  }\KeywordTok{filter}\NormalTok{(year4 }\OperatorTok\StringTok{ }\KeywordTok{c}\NormalTok{(}\StringTok{"1993"}\NormalTok{, }\StringTok{"1994"}\NormalTok{, }\StringTok{"1995"}\NormalTok{, }\StringTok{"1996"}\NormalTok{)) }\OperatorTok
\StringTok{  }\KeywordTok{filter}\NormalTok{(depth_id }\OperatorTok{==}\StringTok{ }\DecValTok{1}\NormalTok{)}

\CommentTok{#change to month to factor}
\NormalTok{lake_nutrients_depth}\OperatorTok{$}\NormalTok{month <-}\StringTok{ }\KeywordTok{as.factor}\NormalTok{(lake_nutrients_depth}\OperatorTok{$}\NormalTok{month)}
\end{Highlighting}
\end{Shaded}

\hypertarget{analysis}{%
\subsection{Analysis}\label{analysis}}

Peter Lake was manipulated with additions of nitrogen and phosphorus
over the years 1993-1996 in an effort to assess the impacts of
eutrophication in lakes. You are tasked with finding out if nutrients
are significantly higher in Peter Lake than Paul Lake, and if these
potential differences in nutrients vary seasonally (use month as a
factor to represent seasonality). Run two separate tests for TN and TP.

\begin{enumerate}
\def\labelenumi{\arabic{enumi}.}
\setcounter{enumi}{3}
\tightlist
\item
  Which application of the GLM will you use (t-test, one-way ANOVA,
  two-way ANOVA with main effects, or two-way ANOVA with interaction
  effects)? Justify your choice.
\end{enumerate}

\begin{quote}
Answer: I choose to run a Two-Way ANOVA with interaction effects. An
ANOVA is being used because there is three or more groups within the
seasons (months) so a T-Test would not be appropriate. A two-way ANOVA
is being used because we want to examine the effects of two categorical
explanatory variables on a continuous varible. Lake and Month are the
two categorical variables being tested on total nitrogen and total
phosphorus which are continous variables. Interaction effects is
included in the two-way ANOVA because the effects of lake and month may
be dependent on each other especially since Peter Lake was manipulated
with nutrients over our time period. A two-way ANOVA with interaction
effects will examine the individual effects of the explanatory variables
as well as the interaction of the explanatory variables.
\end{quote}

\begin{enumerate}
\def\labelenumi{\arabic{enumi}.}
\setcounter{enumi}{4}
\item
  Run your test for TN. Include examination of groupings and consider
  interaction effects, if relevant.
\item
  Run your test for TP. Include examination of groupings and consider
  interaction effects, if relevant.
\end{enumerate}

\begin{Shaded}
\begin{Highlighting}[]
\CommentTok{#5}
\NormalTok{TN_interaction <-}\StringTok{ }\KeywordTok{aov}\NormalTok{(}\DataTypeTok{data =}\NormalTok{ lake_nutrients_depth, tn_ug }\OperatorTok{~}\StringTok{ }\NormalTok{month }\OperatorTok{*}\StringTok{ }\NormalTok{lakename)}
\KeywordTok{summary}\NormalTok{(TN_interaction)}
\end{Highlighting}
\end{Shaded}

\begin{verbatim}
##                Df  Sum Sq Mean Sq F value   Pr(>F)    
## month           4  429686  107421   1.585    0.185    
## lakename        1 2498451 2498451  36.855 2.47e-08 ***
## month:lakename  4  288272   72068   1.063    0.379    
## Residuals      97 6575834   67792                     
## ---
## Signif. codes:  0 '***' 0.001 '**' 0.01 '*' 0.05 '.' 0.1 ' ' 1
## 23 observations deleted due to missingness
\end{verbatim}

\begin{Shaded}
\begin{Highlighting}[]
\CommentTok{#There is no significant interaction effect on month:lakename; P value = 0.379}

\NormalTok{TN_interaction_lm <-}\StringTok{ }\KeywordTok{lm}\NormalTok{(}\DataTypeTok{data =}\NormalTok{ lake_nutrients_depth, tn_ug }\OperatorTok{~}\StringTok{ }\NormalTok{month }\OperatorTok{*}\StringTok{ }\NormalTok{lakename)}
\KeywordTok{summary}\NormalTok{(TN_interaction_lm)}
\end{Highlighting}
\end{Shaded}

\begin{verbatim}
## 
## Call:
## lm(formula = tn_ug ~ month * lakename, data = lake_nutrients_depth)
## 
## Residuals:
##     Min      1Q  Median      3Q     Max 
## -357.88 -118.10  -10.41   50.58 1353.86 
## 
## Coefficients:
##                           Estimate Std. Error t value Pr(>|t|)   
## (Intercept)                 300.51     106.30   2.827   0.0057 **
## month6                       23.61     123.64   0.191   0.8489   
## month7                       53.12     127.05   0.418   0.6768   
## month8                       36.00     127.05   0.283   0.7775   
## month9                      105.82     184.11   0.575   0.5668   
## lakenamePeter Lake           84.43     144.86   0.583   0.5614   
## month6:lakenamePeter Lake   200.49     170.90   1.173   0.2436   
## month7:lakenamePeter Lake   271.82     176.18   1.543   0.1261   
## month8:lakenamePeter Lake   325.05     174.20   1.866   0.0651 . 
## month9:lakenamePeter Lake    59.70     278.35   0.214   0.8306   
## ---
## Signif. codes:  0 '***' 0.001 '**' 0.01 '*' 0.05 '.' 0.1 ' ' 1
## 
## Residual standard error: 260.4 on 97 degrees of freedom
##   (23 observations deleted due to missingness)
## Multiple R-squared:  0.3285, Adjusted R-squared:  0.2662 
## F-statistic: 5.272 on 9 and 97 DF,  p-value: 7.729e-06
\end{verbatim}

\begin{Shaded}
\begin{Highlighting}[]
\CommentTok{#Peter lake has higher total nitrogen than Paul Lake and do not vary seasonally }
\CommentTok{#no further tests need to be run }

\CommentTok{#6}
\NormalTok{TP_interaction <-}\StringTok{ }\KeywordTok{aov}\NormalTok{(}\DataTypeTok{data =}\NormalTok{ lake_nutrients_depth, tp_ug }\OperatorTok{~}\StringTok{ }\NormalTok{month }\OperatorTok{*}\StringTok{ }\NormalTok{lakename)}
\KeywordTok{summary}\NormalTok{(TP_interaction)}
\end{Highlighting}
\end{Shaded}

\begin{verbatim}
##                 Df Sum Sq Mean Sq F value Pr(>F)    
## month            4    671     168   1.623 0.1730    
## lakename         1  10370   10370 100.283 <2e-16 ***
## month:lakename   4   1014     254   2.452 0.0496 *  
## Residuals      119  12305     103                   
## ---
## Signif. codes:  0 '***' 0.001 '**' 0.01 '*' 0.05 '.' 0.1 ' ' 1
## 1 observation deleted due to missingness
\end{verbatim}

\begin{Shaded}
\begin{Highlighting}[]
\CommentTok{#There is a significant interaction effect --> interpret interaction effects only}

\NormalTok{TP_interaction_lm <-}\StringTok{ }\KeywordTok{lm}\NormalTok{(}\DataTypeTok{data =}\NormalTok{ lake_nutrients_depth, tp_ug }\OperatorTok{~}\StringTok{ }\NormalTok{month }\OperatorTok{*}\StringTok{ }\NormalTok{lakename)}
\KeywordTok{summary}\NormalTok{(TP_interaction_lm)}
\end{Highlighting}
\end{Shaded}

\begin{verbatim}
## 
## Call:
## lm(formula = tp_ug ~ month * lakename, data = lake_nutrients_depth)
## 
## Residuals:
##     Min      1Q  Median      3Q     Max 
## -17.384  -4.473  -0.693   1.939  32.489 
## 
## Coefficients:
##                           Estimate Std. Error t value Pr(>|t|)   
## (Intercept)                11.4740     4.1514   2.764  0.00662 **
## month6                     -0.9179     4.8288  -0.190  0.84957   
## month7                     -1.7271     4.7936  -0.360  0.71927   
## month8                     -2.0872     4.7936  -0.435  0.66405   
## month9                     -0.7380     6.1575  -0.120  0.90480   
## lakenamePeter Lake          4.3136     5.6574   0.762  0.44729   
## month6:lakenamePeter Lake  13.4882     6.6207   2.037  0.04384 * 
## month7:lakenamePeter Lake  20.3440     6.6207   3.073  0.00263 **
## month8:lakenamePeter Lake  12.7937     6.5722   1.947  0.05394 . 
## month9:lakenamePeter Lake  11.1697     8.8622   1.260  0.21000   
## ---
## Signif. codes:  0 '***' 0.001 '**' 0.01 '*' 0.05 '.' 0.1 ' ' 1
## 
## Residual standard error: 10.17 on 119 degrees of freedom
##   (1 observation deleted due to missingness)
## Multiple R-squared:  0.4949, Adjusted R-squared:  0.4567 
## F-statistic: 12.95 on 9 and 119 DF,  p-value: 3.24e-14
\end{verbatim}

\begin{Shaded}
\begin{Highlighting}[]
\CommentTok{#Peter lake has higher total phosphorus than Paul Lake}

\CommentTok{#Run a post-hoc test for pairwise differences}
\KeywordTok{TukeyHSD}\NormalTok{(TP_interaction)}
\end{Highlighting}
\end{Shaded}

\begin{verbatim}
##   Tukey multiple comparisons of means
##     95% family-wise confidence level
## 
## Fit: aov(formula = tp_ug ~ month * lakename, data = lake_nutrients_depth)
## 
## $month
##           diff        lwr       upr     p adj
## 6-5  5.9146220  -3.234390 15.063634 0.3837749
## 7-5  7.9267363  -1.222276 17.075748 0.1224572
## 8-5  4.3748753  -4.706921 13.456671 0.6703911
## 9-5  3.8207521  -8.393804 16.035308 0.9085595
## 7-6  2.0121143  -4.721376  8.745605 0.9215444
## 8-6 -1.5397467  -8.181621  5.102128 0.9677800
## 9-6 -2.0938698 -12.621493  8.433754 0.9816312
## 8-7 -3.5518610 -10.193735  3.090013 0.5765788
## 9-7 -4.1059841 -14.633608  6.421639 0.8162959
## 9-8 -0.5541231 -11.023385  9.915139 0.9998946
## 
## $lakename
##                          diff      lwr      upr p adj
## Peter Lake-Paul Lake 17.91381 14.36807 21.45955     0
## 
## $`month:lakename`
##                                 diff         lwr       upr     p adj
## 6:Paul Lake-5:Paul Lake   -0.9178824 -16.4886641 14.652899 1.0000000
## 7:Paul Lake-5:Paul Lake   -1.7271111 -17.1846493 13.730427 0.9999981
## 8:Paul Lake-5:Paul Lake   -2.0872222 -17.5447604 13.370316 0.9999902
## 9:Paul Lake-5:Paul Lake   -0.7380000 -20.5935673 19.117567 1.0000000
## 5:Peter Lake-5:Paul Lake   4.3135714 -13.9293175 22.556460 0.9989515
## 6:Peter Lake-5:Paul Lake  16.8838889   1.4263507 32.341427 0.0206973
## 7:Peter Lake-5:Paul Lake  22.9304706   7.3596889 38.501252 0.0002415
## 8:Peter Lake-5:Paul Lake  15.0200000  -0.3355071 30.375507 0.0607728
## 9:Peter Lake-5:Paul Lake  14.7452500  -6.4208558 35.911356 0.4316694
## 7:Paul Lake-6:Paul Lake   -0.8092288 -11.8989312 10.280474 1.0000000
## 8:Paul Lake-6:Paul Lake   -1.1693399 -12.2590423  9.920363 0.9999989
## 9:Paul Lake-6:Paul Lake    0.1798824 -16.5021309 16.861896 1.0000000
## 5:Peter Lake-6:Paul Lake   5.2314538  -9.4943403 19.957248 0.9787107
## 6:Peter Lake-6:Paul Lake  17.8017712   6.7120688 28.891474 0.0000401
## 7:Peter Lake-6:Paul Lake  23.8483529  12.6013419 35.095364 0.0000000
## 8:Peter Lake-6:Paul Lake  15.9378824   4.9908457 26.884919 0.0003006
## 9:Peter Lake-6:Paul Lake  15.6631324  -2.5591082 33.885373 0.1584032
## 8:Paul Lake-7:Paul Lake   -0.3601111 -11.2902412 10.570019 1.0000000
## 9:Paul Lake-7:Paul Lake    0.9891111 -15.5872518 17.565474 1.0000000
## 5:Peter Lake-7:Paul Lake   6.0406825  -8.5653181 20.646683 0.9437275
## 6:Peter Lake-7:Paul Lake  18.6110000   7.6808700 29.541130 0.0000101
## 7:Peter Lake-7:Paul Lake  24.6575817  13.5678793 35.747284 0.0000000
## 8:Peter Lake-7:Paul Lake  16.7471111   5.9617574 27.532465 0.0000827
## 9:Peter Lake-7:Paul Lake  16.4723611  -1.6532090 34.597931 0.1087387
## 9:Paul Lake-8:Paul Lake    1.3492222 -15.2271407 17.925585 0.9999999
## 5:Peter Lake-8:Paul Lake   6.4007937  -8.2052070 21.006794 0.9208652
## 6:Peter Lake-8:Paul Lake  18.9711111   8.0409811 29.901241 0.0000062
## 7:Peter Lake-8:Paul Lake  25.0176928  13.9279904 36.107395 0.0000000
## 8:Peter Lake-8:Paul Lake  17.1072222   6.3218685 27.892576 0.0000523
## 9:Peter Lake-8:Paul Lake  16.8324722  -1.2930979 34.958042 0.0926020
## 5:Peter Lake-9:Paul Lake   5.0515714 -14.1485150 24.251658 0.9975850
## 6:Peter Lake-9:Paul Lake  17.6218889   1.0455259 34.198252 0.0276305
## 7:Peter Lake-9:Paul Lake  23.6684706   6.9864574 40.350484 0.0004851
## 8:Peter Lake-9:Paul Lake  15.7580000  -0.7232597 32.239260 0.0735733
## 9:Peter Lake-9:Paul Lake  15.4832500  -6.5132124 37.479712 0.4163366
## 6:Peter Lake-5:Peter Lake 12.5703175  -2.0356832 27.176318 0.1571717
## 7:Peter Lake-5:Peter Lake 18.6168992   3.8911050 33.342693 0.0032014
## 8:Peter Lake-5:Peter Lake 10.7064286  -3.7915495 25.204407 0.3464892
## 9:Peter Lake-5:Peter Lake 10.4316786 -10.1207861 30.984143 0.8273658
## 7:Peter Lake-6:Peter Lake  6.0465817  -5.0431207 17.136284 0.7595330
## 8:Peter Lake-6:Peter Lake -1.8638889 -12.6492426  8.921465 0.9999197
## 9:Peter Lake-6:Peter Lake -2.1386389 -20.2642090 15.986931 0.9999970
## 8:Peter Lake-7:Peter Lake -7.9104706 -18.8575073  3.036566 0.3778093
## 9:Peter Lake-7:Peter Lake -8.1852206 -26.4074611 10.037020 0.9089776
## 9:Peter Lake-8:Peter Lake -0.2747500 -18.3133864 17.763886 1.0000000
\end{verbatim}

\begin{Shaded}
\begin{Highlighting}[]
\NormalTok{TP_interaction.lake.month <-}\StringTok{ }\KeywordTok{with}\NormalTok{(lake_nutrients_depth, }
                                  \KeywordTok{interaction}\NormalTok{(month, lakename))}
\NormalTok{TP_interaction.lake.month.aov <-}\StringTok{ }\KeywordTok{aov}\NormalTok{(}\DataTypeTok{data =}\NormalTok{ lake_nutrients_depth, }
\NormalTok{                                     tp_ug }\OperatorTok{~}\StringTok{ }\NormalTok{TP_interaction.lake.month)}

\NormalTok{TP_interaction_groups <-}\StringTok{ }\KeywordTok{HSD.test}\NormalTok{(TP_interaction.lake.month.aov, }
                                  \StringTok{"TP_interaction.lake.month"}\NormalTok{, }\DataTypeTok{group =} \OtherTok{TRUE}\NormalTok{)}
\NormalTok{TP_interaction_groups}
\end{Highlighting}
\end{Shaded}

\begin{verbatim}
## $statistics
##    MSerror  Df     Mean      CV
##   103.4055 119 19.07347 53.3141
## 
## $parameters
##    test                    name.t ntr StudentizedRange alpha
##   Tukey TP_interaction.lake.month  10         4.560262  0.05
## 
## $means
##                  tp_ug       std  r    Min    Max     Q25     Q50      Q75
## 5.Paul Lake  11.474000  3.928545  6  7.001 17.090  8.1395 11.8885 13.53675
## 5.Peter Lake 15.787571  2.719954  7 10.887 18.922 14.8915 15.5730 17.67400
## 6.Paul Lake  10.556118  4.416821 17  1.222 16.697  7.4430 10.6050 13.94600
## 6.Peter Lake 28.357889 15.588507 18 10.974 53.388 14.7790 24.6840 41.13000
## 7.Paul Lake   9.746889  3.525120 18  4.501 21.763  7.8065  9.1555 10.65700
## 7.Peter Lake 34.404471 18.285568 17 19.149 66.893 21.6640 24.2070 50.54900
## 8.Paul Lake   9.386778  1.478062 18  5.879 11.542  8.4495  9.6090 10.45050
## 8.Peter Lake 26.494000  9.829596 19 14.551 49.757 21.2425 23.2250 27.99350
## 9.Paul Lake  10.736000  3.615978  5  6.592 16.281  8.9440 10.1920 11.67100
## 9.Peter Lake 26.219250 10.814803  4 16.281 41.145 19.6845 23.7255 30.26025
## 
## $comparison
## NULL
## 
## $groups
##                  tp_ug groups
## 7.Peter Lake 34.404471      a
## 6.Peter Lake 28.357889     ab
## 8.Peter Lake 26.494000    abc
## 9.Peter Lake 26.219250   abcd
## 5.Peter Lake 15.787571    bcd
## 5.Paul Lake  11.474000     cd
## 9.Paul Lake  10.736000     cd
## 6.Paul Lake  10.556118      d
## 7.Paul Lake   9.746889      d
## 8.Paul Lake   9.386778      d
## 
## attr(,"class")
## [1] "group"
\end{verbatim}

\begin{enumerate}
\def\labelenumi{\arabic{enumi}.}
\setcounter{enumi}{6}
\item
  Create two plots, with TN (plot 1) or TP (plot 2) as the response
  variable and month and lake as the predictor variables. Hint: you may
  use some of the code you used for your visualization assignment.
  Assign groupings with letters, as determined from your tests. Adjust
  your axes, aesthetics, and color palettes in accordance with best data
  visualization practices.
\item
  Combine your plots with cowplot, with a common legend at the top and
  the two graphs stacked vertically. Your x axes should be formatted
  with the same breaks, such that you can remove the title and text of
  the top legend and retain just the bottom legend.
\end{enumerate}

\begin{Shaded}
\begin{Highlighting}[]
\NormalTok{mytheme <-}\StringTok{ }\KeywordTok{theme_stata}\NormalTok{(}\DataTypeTok{base_size =} \DecValTok{14}\NormalTok{, }\DataTypeTok{base_family =} \StringTok{"sans"}\NormalTok{, }\DataTypeTok{scheme =} \StringTok{"s2mono"}\NormalTok{) }\OperatorTok{+}
\StringTok{  }\KeywordTok{theme}\NormalTok{(}\DataTypeTok{axis.text =} \KeywordTok{element_text}\NormalTok{(}\DataTypeTok{color =} \StringTok{"black"}\NormalTok{), }
        \DataTypeTok{legend.position =} \StringTok{"top"}\NormalTok{)}

\KeywordTok{theme_set}\NormalTok{(mytheme)}

\CommentTok{#7}
\CommentTok{#Total Nitrogen Plot}
\NormalTok{TN_plot <-}\StringTok{ }\KeywordTok{ggplot}\NormalTok{(lake_nutrients_depth, }\KeywordTok{aes}\NormalTok{(}\DataTypeTok{x =}\NormalTok{ month, }\DataTypeTok{y =}\NormalTok{ tn_ug, }\DataTypeTok{color =}\NormalTok{ lakename)) }\OperatorTok{+}
\StringTok{  }\KeywordTok{geom_boxplot}\NormalTok{() }\OperatorTok{+}
\StringTok{  }\KeywordTok{labs}\NormalTok{(}\DataTypeTok{x =} \StringTok{"Month"}\NormalTok{, }\DataTypeTok{y =} \KeywordTok{expression}\NormalTok{(}\KeywordTok{paste}\NormalTok{(}\StringTok{"Total Nitrogen ("}\NormalTok{, mu, }\StringTok{"g/L)"}\NormalTok{)), }
       \DataTypeTok{color =} \StringTok{"Lake Name"}\NormalTok{) }\OperatorTok{+}
\StringTok{  }\KeywordTok{scale_color_stata}\NormalTok{(}\StringTok{"s2color"}\NormalTok{) }\OperatorTok{+}
\StringTok{  }\KeywordTok{ylim}\NormalTok{(}\DecValTok{0}\NormalTok{,}\DecValTok{2100}\NormalTok{)}

\KeywordTok{print}\NormalTok{(TN_plot)}
\end{Highlighting}
\end{Shaded}

\begin{verbatim}
## Warning: Removed 23 rows containing non-finite values (stat_boxplot).
\end{verbatim}

\includegraphics{A06_GLMs_Week1_files/figure-latex/unnamed-chunk-4-1.pdf}

\begin{Shaded}
\begin{Highlighting}[]
\CommentTok{#Total Phosphorus Plot}
\NormalTok{TP_plot <-}\StringTok{ }\KeywordTok{ggplot}\NormalTok{(lake_nutrients_depth, }\KeywordTok{aes}\NormalTok{(}\DataTypeTok{x =}\NormalTok{ month, }\DataTypeTok{y =}\NormalTok{ tp_ug, }\DataTypeTok{color =}\NormalTok{ lakename)) }\OperatorTok{+}
\StringTok{  }\KeywordTok{geom_boxplot}\NormalTok{() }\OperatorTok{+}
\StringTok{  }\KeywordTok{labs}\NormalTok{(}\DataTypeTok{x =} \StringTok{"Month"}\NormalTok{, }\DataTypeTok{y =} \KeywordTok{expression}\NormalTok{(}\KeywordTok{paste}\NormalTok{(}\StringTok{"Total Phosphorus("}\NormalTok{, mu, }\StringTok{"g/L)"}\NormalTok{)), }
       \DataTypeTok{color =} \StringTok{"Lake Name"}\NormalTok{) }\OperatorTok{+}
\StringTok{  }\KeywordTok{scale_color_stata}\NormalTok{(}\StringTok{"s2color"}\NormalTok{) }\OperatorTok{+}
\StringTok{  }\KeywordTok{stat_summary}\NormalTok{(}\DataTypeTok{geom =} \StringTok{"text"}\NormalTok{, }\DataTypeTok{fun.y =}\NormalTok{ max, }\DataTypeTok{vjust =} \DecValTok{-1}\NormalTok{, }\DataTypeTok{size =} \FloatTok{3.5}\NormalTok{,}
               \DataTypeTok{label =} \KeywordTok{c}\NormalTok{(}\StringTok{"cd"}\NormalTok{, }\StringTok{"bcd"}\NormalTok{, }\StringTok{"d"}\NormalTok{, }\StringTok{"ab"}\NormalTok{, }\StringTok{"d"}\NormalTok{, }\StringTok{"a"}\NormalTok{, }
                         \StringTok{"d"}\NormalTok{, }\StringTok{"abc"}\NormalTok{, }\StringTok{"cd"}\NormalTok{, }\StringTok{"abcd"}\NormalTok{),}
               \DataTypeTok{show.legend =} \OtherTok{FALSE}\NormalTok{,}
               \DataTypeTok{position =} \KeywordTok{position_dodge2}\NormalTok{(}\FloatTok{0.6}\NormalTok{)) }\OperatorTok{+}
\StringTok{  }\KeywordTok{ylim}\NormalTok{(}\DecValTok{0}\NormalTok{,}\DecValTok{70}\NormalTok{)}

\KeywordTok{print}\NormalTok{(TP_plot)}
\end{Highlighting}
\end{Shaded}

\begin{verbatim}
## Warning: Removed 1 rows containing non-finite values (stat_boxplot).
\end{verbatim}

\begin{verbatim}
## Warning: Removed 1 rows containing non-finite values (stat_summary).
\end{verbatim}

\includegraphics{A06_GLMs_Week1_files/figure-latex/unnamed-chunk-4-2.pdf}

\begin{Shaded}
\begin{Highlighting}[]
\CommentTok{#8 }
\NormalTok{boxplots_nutrients <-}\StringTok{ }
\StringTok{  }\KeywordTok{plot_grid}\NormalTok{(TN_plot }\OperatorTok{+}\StringTok{ }\KeywordTok{theme}\NormalTok{(}\DataTypeTok{axis.text.x =} \KeywordTok{element_blank}\NormalTok{(),}
                            \DataTypeTok{axis.ticks.x =} \KeywordTok{element_blank}\NormalTok{(),}
                            \DataTypeTok{axis.title.x =} \KeywordTok{element_blank}\NormalTok{()),}
\NormalTok{            TP_plot }\OperatorTok{+}\StringTok{ }\KeywordTok{theme}\NormalTok{(}\DataTypeTok{legend.position =} \StringTok{"none"}\NormalTok{),}
            \DataTypeTok{labels =} \StringTok{"AUTO"}\NormalTok{, }
            \DataTypeTok{align =} \StringTok{"h"}\NormalTok{, }
            \DataTypeTok{axis =} \StringTok{"b"}\NormalTok{,}
            \DataTypeTok{ncol =} \DecValTok{1}\NormalTok{)}
\end{Highlighting}
\end{Shaded}

\begin{verbatim}
## Warning: Removed 23 rows containing non-finite values (stat_boxplot).
\end{verbatim}

\begin{verbatim}
## Warning: Removed 1 rows containing non-finite values (stat_boxplot).
\end{verbatim}

\begin{verbatim}
## Warning: Removed 1 rows containing non-finite values (stat_summary).
\end{verbatim}

\begin{Shaded}
\begin{Highlighting}[]
\KeywordTok{print}\NormalTok{(boxplots_nutrients)}
\end{Highlighting}
\end{Shaded}

\includegraphics{A06_GLMs_Week1_files/figure-latex/unnamed-chunk-4-3.pdf}


\end{document}
