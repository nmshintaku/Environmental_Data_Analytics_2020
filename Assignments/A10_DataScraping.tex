\documentclass[]{article}
\usepackage{lmodern}
\usepackage{amssymb,amsmath}
\usepackage{ifxetex,ifluatex}
\usepackage{fixltx2e} % provides \textsubscript
\ifnum 0\ifxetex 1\fi\ifluatex 1\fi=0 % if pdftex
  \usepackage[T1]{fontenc}
  \usepackage[utf8]{inputenc}
\else % if luatex or xelatex
  \ifxetex
    \usepackage{mathspec}
  \else
    \usepackage{fontspec}
  \fi
  \defaultfontfeatures{Ligatures=TeX,Scale=MatchLowercase}
\fi
% use upquote if available, for straight quotes in verbatim environments
\IfFileExists{upquote.sty}{\usepackage{upquote}}{}
% use microtype if available
\IfFileExists{microtype.sty}{%
\usepackage{microtype}
\UseMicrotypeSet[protrusion]{basicmath} % disable protrusion for tt fonts
}{}
\usepackage[margin=2.54cm]{geometry}
\usepackage{hyperref}
\hypersetup{unicode=true,
            pdftitle={Assignment 10: Data Scraping},
            pdfauthor={Nikki Shintaku},
            pdfborder={0 0 0},
            breaklinks=true}
\urlstyle{same}  % don't use monospace font for urls
\usepackage{color}
\usepackage{fancyvrb}
\newcommand{\VerbBar}{|}
\newcommand{\VERB}{\Verb[commandchars=\\\{\}]}
\DefineVerbatimEnvironment{Highlighting}{Verbatim}{commandchars=\\\{\}}
% Add ',fontsize=\small' for more characters per line
\usepackage{framed}
\definecolor{shadecolor}{RGB}{248,248,248}
\newenvironment{Shaded}{\begin{snugshade}}{\end{snugshade}}
\newcommand{\AlertTok}[1]{\textcolor[rgb]{0.94,0.16,0.16}{#1}}
\newcommand{\AnnotationTok}[1]{\textcolor[rgb]{0.56,0.35,0.01}{\textbf{\textit{#1}}}}
\newcommand{\AttributeTok}[1]{\textcolor[rgb]{0.77,0.63,0.00}{#1}}
\newcommand{\BaseNTok}[1]{\textcolor[rgb]{0.00,0.00,0.81}{#1}}
\newcommand{\BuiltInTok}[1]{#1}
\newcommand{\CharTok}[1]{\textcolor[rgb]{0.31,0.60,0.02}{#1}}
\newcommand{\CommentTok}[1]{\textcolor[rgb]{0.56,0.35,0.01}{\textit{#1}}}
\newcommand{\CommentVarTok}[1]{\textcolor[rgb]{0.56,0.35,0.01}{\textbf{\textit{#1}}}}
\newcommand{\ConstantTok}[1]{\textcolor[rgb]{0.00,0.00,0.00}{#1}}
\newcommand{\ControlFlowTok}[1]{\textcolor[rgb]{0.13,0.29,0.53}{\textbf{#1}}}
\newcommand{\DataTypeTok}[1]{\textcolor[rgb]{0.13,0.29,0.53}{#1}}
\newcommand{\DecValTok}[1]{\textcolor[rgb]{0.00,0.00,0.81}{#1}}
\newcommand{\DocumentationTok}[1]{\textcolor[rgb]{0.56,0.35,0.01}{\textbf{\textit{#1}}}}
\newcommand{\ErrorTok}[1]{\textcolor[rgb]{0.64,0.00,0.00}{\textbf{#1}}}
\newcommand{\ExtensionTok}[1]{#1}
\newcommand{\FloatTok}[1]{\textcolor[rgb]{0.00,0.00,0.81}{#1}}
\newcommand{\FunctionTok}[1]{\textcolor[rgb]{0.00,0.00,0.00}{#1}}
\newcommand{\ImportTok}[1]{#1}
\newcommand{\InformationTok}[1]{\textcolor[rgb]{0.56,0.35,0.01}{\textbf{\textit{#1}}}}
\newcommand{\KeywordTok}[1]{\textcolor[rgb]{0.13,0.29,0.53}{\textbf{#1}}}
\newcommand{\NormalTok}[1]{#1}
\newcommand{\OperatorTok}[1]{\textcolor[rgb]{0.81,0.36,0.00}{\textbf{#1}}}
\newcommand{\OtherTok}[1]{\textcolor[rgb]{0.56,0.35,0.01}{#1}}
\newcommand{\PreprocessorTok}[1]{\textcolor[rgb]{0.56,0.35,0.01}{\textit{#1}}}
\newcommand{\RegionMarkerTok}[1]{#1}
\newcommand{\SpecialCharTok}[1]{\textcolor[rgb]{0.00,0.00,0.00}{#1}}
\newcommand{\SpecialStringTok}[1]{\textcolor[rgb]{0.31,0.60,0.02}{#1}}
\newcommand{\StringTok}[1]{\textcolor[rgb]{0.31,0.60,0.02}{#1}}
\newcommand{\VariableTok}[1]{\textcolor[rgb]{0.00,0.00,0.00}{#1}}
\newcommand{\VerbatimStringTok}[1]{\textcolor[rgb]{0.31,0.60,0.02}{#1}}
\newcommand{\WarningTok}[1]{\textcolor[rgb]{0.56,0.35,0.01}{\textbf{\textit{#1}}}}
\usepackage{graphicx,grffile}
\makeatletter
\def\maxwidth{\ifdim\Gin@nat@width>\linewidth\linewidth\else\Gin@nat@width\fi}
\def\maxheight{\ifdim\Gin@nat@height>\textheight\textheight\else\Gin@nat@height\fi}
\makeatother
% Scale images if necessary, so that they will not overflow the page
% margins by default, and it is still possible to overwrite the defaults
% using explicit options in \includegraphics[width, height, ...]{}
\setkeys{Gin}{width=\maxwidth,height=\maxheight,keepaspectratio}
\IfFileExists{parskip.sty}{%
\usepackage{parskip}
}{% else
\setlength{\parindent}{0pt}
\setlength{\parskip}{6pt plus 2pt minus 1pt}
}
\setlength{\emergencystretch}{3em}  % prevent overfull lines
\providecommand{\tightlist}{%
  \setlength{\itemsep}{0pt}\setlength{\parskip}{0pt}}
\setcounter{secnumdepth}{0}
% Redefines (sub)paragraphs to behave more like sections
\ifx\paragraph\undefined\else
\let\oldparagraph\paragraph
\renewcommand{\paragraph}[1]{\oldparagraph{#1}\mbox{}}
\fi
\ifx\subparagraph\undefined\else
\let\oldsubparagraph\subparagraph
\renewcommand{\subparagraph}[1]{\oldsubparagraph{#1}\mbox{}}
\fi

%%% Use protect on footnotes to avoid problems with footnotes in titles
\let\rmarkdownfootnote\footnote%
\def\footnote{\protect\rmarkdownfootnote}

%%% Change title format to be more compact
\usepackage{titling}

% Create subtitle command for use in maketitle
\providecommand{\subtitle}[1]{
  \posttitle{
    \begin{center}\large#1\end{center}
    }
}

\setlength{\droptitle}{-2em}

  \title{Assignment 10: Data Scraping}
    \pretitle{\vspace{\droptitle}\centering\huge}
  \posttitle{\par}
    \author{Nikki Shintaku}
    \preauthor{\centering\large\emph}
  \postauthor{\par}
    \date{}
    \predate{}\postdate{}
  

\begin{document}
\maketitle

\hypertarget{total-points}{%
\section{Total points:}\label{total-points}}

\hypertarget{overview}{%
\subsection{OVERVIEW}\label{overview}}

This exercise accompanies the lessons in Environmental Data Analytics on
time series analysis.

\hypertarget{directions}{%
\subsection{Directions}\label{directions}}

\begin{enumerate}
\def\labelenumi{\arabic{enumi}.}
\tightlist
\item
  Change ``Student Name'' on line 3 (above) with your name.
\item
  Work through the steps, \textbf{creating code and output} that fulfill
  each instruction.
\item
  Be sure to \textbf{answer the questions} in this assignment document.
\item
  When you have completed the assignment, \textbf{Knit} the text and
  code into a single PDF file.
\item
  After Knitting, submit the completed exercise (PDF file) to the
  dropbox in Sakai. Add your last name into the file name (e.g.,
  ``Salk\_A06\_GLMs\_Week1.Rmd'') prior to submission.
\end{enumerate}

The completed exercise is due on Tuesday, April 7 at 1:00 pm.

\hypertarget{set-up}{%
\subsection{Set up}\label{set-up}}

\begin{enumerate}
\def\labelenumi{\arabic{enumi}.}
\tightlist
\item
  Set up your session:
\end{enumerate}

\begin{itemize}
\tightlist
\item
  Check your working directory
\item
  Load the packages \texttt{tidyverse}, \texttt{rvest}, and any others
  you end up using.
\item
  Set your ggplot theme
\end{itemize}

\begin{Shaded}
\begin{Highlighting}[]
\KeywordTok{getwd}\NormalTok{()}
\end{Highlighting}
\end{Shaded}

\begin{verbatim}
## [1] "/Users/nikkishintaku/Desktop/Environmental872/Environmental_Data_Analytics_2020"
\end{verbatim}

\begin{Shaded}
\begin{Highlighting}[]
\KeywordTok{library}\NormalTok{(tidyverse)}
\KeywordTok{library}\NormalTok{(viridis)}
\KeywordTok{library}\NormalTok{(rvest)}
\KeywordTok{library}\NormalTok{(ggrepel)}
\KeywordTok{library}\NormalTok{(ggthemes)}

\NormalTok{mytheme <-}\StringTok{ }\KeywordTok{theme_stata}\NormalTok{(}\DataTypeTok{base_size =} \DecValTok{14}\NormalTok{, }\DataTypeTok{base_family =} \StringTok{"sans"}\NormalTok{, }\DataTypeTok{scheme =} \StringTok{"s2mono"}\NormalTok{) }\OperatorTok{+}
\StringTok{  }\KeywordTok{theme}\NormalTok{(}\DataTypeTok{axis.text =} \KeywordTok{element_text}\NormalTok{(}\DataTypeTok{color =} \StringTok{"black"}\NormalTok{), }
        \DataTypeTok{legend.position =} \StringTok{"top"}\NormalTok{)}

\KeywordTok{theme_set}\NormalTok{(mytheme)}
\end{Highlighting}
\end{Shaded}

\begin{enumerate}
\def\labelenumi{\arabic{enumi}.}
\setcounter{enumi}{1}
\tightlist
\item
  Indicate the EPA impaired waters website
  (\url{https://www.epa.gov/nutrient-policy-data/waters-assessed-impaired-due-nutrient-related-causes})
  as the URL to be scraped.
\end{enumerate}

\begin{Shaded}
\begin{Highlighting}[]
\CommentTok{#url for data scraping}
\NormalTok{url <-}\StringTok{ "https://www.epa.gov/nutrient-policy-data/waters-assessed-impaired-due-nutrient-related-causes"}

\CommentTok{# Reading the HTML code from the website}
\NormalTok{webpage <-}\StringTok{ }\KeywordTok{read_html}\NormalTok{(url)}
\end{Highlighting}
\end{Shaded}

\begin{enumerate}
\def\labelenumi{\arabic{enumi}.}
\setcounter{enumi}{2}
\tightlist
\item
  Scrape the Rivers table, with every column except year. Then, turn it
  into a data frame.
\end{enumerate}

\begin{Shaded}
\begin{Highlighting}[]
\NormalTok{State <-}\StringTok{ }\NormalTok{webpage }\OperatorTok\StringTok{ }\KeywordTok{html_nodes}\NormalTok{(}\StringTok{"table:nth-child(8) td:nth-child(1)"}\NormalTok{) }\OperatorTok\StringTok{ }\KeywordTok{html_text}\NormalTok{()}
\NormalTok{Rivers.Assessed.mi2 <-}\StringTok{ }\NormalTok{webpage }\OperatorTok\StringTok{ }\KeywordTok{html_nodes}\NormalTok{(}\StringTok{"table:nth-child(8) td:nth-child(2)"}\NormalTok{) }\OperatorTok\StringTok{ }\KeywordTok{html_text}\NormalTok{()}
\NormalTok{Rivers.Assessed.percent <-}\StringTok{ }\NormalTok{webpage }\OperatorTok\StringTok{ }\KeywordTok{html_nodes}\NormalTok{(}\StringTok{"table:nth-child(8) td:nth-child(3)"}\NormalTok{) }\OperatorTok\StringTok{ }\KeywordTok{html_text}\NormalTok{()}
\NormalTok{Rivers.Impaired.mi2 <-}\StringTok{ }\NormalTok{webpage }\OperatorTok\StringTok{ }\KeywordTok{html_nodes}\NormalTok{(}\StringTok{"table:nth-child(8) td:nth-child(4)"}\NormalTok{) }\OperatorTok\StringTok{ }\KeywordTok{html_text}\NormalTok{()}
\NormalTok{Rivers.Impaired.percent <-}\StringTok{ }\NormalTok{webpage }\OperatorTok\StringTok{ }\KeywordTok{html_nodes}\NormalTok{(}\StringTok{"table:nth-child(8) td:nth-child(5)"}\NormalTok{) }\OperatorTok\StringTok{ }\KeywordTok{html_text}\NormalTok{()}
\NormalTok{Rivers.Impaired.percent.TMDL <-}\StringTok{ }\NormalTok{webpage }\OperatorTok\StringTok{ }\KeywordTok{html_nodes}\NormalTok{(}\StringTok{"table:nth-child(8) td:nth-child(6)"}\NormalTok{) }\OperatorTok\StringTok{ }\KeywordTok{html_text}\NormalTok{()}

\NormalTok{Rivers <-}\StringTok{ }\KeywordTok{data.frame}\NormalTok{(State, Rivers.Assessed.mi2, Rivers.Assessed.percent, }
\NormalTok{                          Rivers.Impaired.mi2, Rivers.Impaired.percent, }
\NormalTok{                          Rivers.Impaired.percent.TMDL)}
\end{Highlighting}
\end{Shaded}

\begin{enumerate}
\def\labelenumi{\arabic{enumi}.}
\setcounter{enumi}{3}
\item
  Use \texttt{str\_replace} to remove non-numeric characters from the
  numeric columns.
\item
  Set the numeric columns to a numeric class and verify this using
  \texttt{str}.
\end{enumerate}

\begin{Shaded}
\begin{Highlighting}[]
\CommentTok{# 4 Remove non-numeric characters}
\NormalTok{Rivers}\OperatorTok{$}\NormalTok{Rivers.Assessed.mi2 <-}\StringTok{ }\KeywordTok{str_replace}\NormalTok{(Rivers}\OperatorTok{$}\NormalTok{Rivers.Assessed.mi2,}
                                                      \DataTypeTok{pattern =} \StringTok{"([,])"}\NormalTok{, }\DataTypeTok{replacement =} \StringTok{""}\NormalTok{)}
\NormalTok{Rivers}\OperatorTok{$}\NormalTok{Rivers.Assessed.percent <-}\StringTok{ }\KeywordTok{str_replace}\NormalTok{(Rivers}\OperatorTok{$}\NormalTok{Rivers.Assessed.percent, }
                                              \DataTypeTok{pattern =} \StringTok{"([%])"}\NormalTok{, }\DataTypeTok{replacement =} \StringTok{""}\NormalTok{)}
\NormalTok{Rivers}\OperatorTok{$}\NormalTok{Rivers.Assessed.percent <-}\KeywordTok{str_replace}\NormalTok{(Rivers}\OperatorTok{$}\NormalTok{Rivers.Assessed.percent,}
                                             \DataTypeTok{pattern =} \StringTok{"([*])"}\NormalTok{, }\DataTypeTok{replacement =} \StringTok{""}\NormalTok{)}
\NormalTok{Rivers}\OperatorTok{$}\NormalTok{Rivers.Impaired.mi2 <-}\StringTok{ }\KeywordTok{str_replace}\NormalTok{(Rivers}\OperatorTok{$}\NormalTok{Rivers.Impaired.mi2,}
                                                      \DataTypeTok{pattern =} \StringTok{"([,])"}\NormalTok{, }\DataTypeTok{replacement =} \StringTok{""}\NormalTok{)}
\NormalTok{Rivers}\OperatorTok{$}\NormalTok{Rivers.Impaired.percent <-}\StringTok{ }\KeywordTok{str_replace}\NormalTok{(Rivers}\OperatorTok{$}\NormalTok{Rivers.Impaired.percent, }
                                              \DataTypeTok{pattern =} \StringTok{"([%])"}\NormalTok{, }\DataTypeTok{replacement =} \StringTok{""}\NormalTok{)}
\NormalTok{Rivers}\OperatorTok{$}\NormalTok{Rivers.Impaired.percent.TMDL <-}\StringTok{ }\KeywordTok{str_replace}\NormalTok{(Rivers}\OperatorTok{$}\NormalTok{Rivers.Impaired.percent.TMDL,}
                                                   \DataTypeTok{pattern =} \StringTok{"([%])"}\NormalTok{, }\DataTypeTok{replacement =} \StringTok{""}\NormalTok{)}
\NormalTok{Rivers}\OperatorTok{$}\NormalTok{Rivers.Impaired.percent.TMDL <-}\StringTok{ }\KeywordTok{str_replace}\NormalTok{(Rivers}\OperatorTok{$}\NormalTok{Rivers.Impaired.percent.TMDL,}
                                                   \DataTypeTok{pattern =} \StringTok{"([±])"}\NormalTok{, }\DataTypeTok{replacement =} \StringTok{""}\NormalTok{)}

\CommentTok{# 5 Set to numeric }
\NormalTok{Rivers}\OperatorTok{$}\NormalTok{Rivers.Assessed.mi2 <-}\StringTok{ }\KeywordTok{as.numeric}\NormalTok{(Rivers}\OperatorTok{$}\NormalTok{Rivers.Assessed.mi2)}
\NormalTok{Rivers}\OperatorTok{$}\NormalTok{Rivers.Assessed.percent <-}\StringTok{ }\KeywordTok{as.numeric}\NormalTok{(Rivers}\OperatorTok{$}\NormalTok{Rivers.Assessed.percent)}
\NormalTok{Rivers}\OperatorTok{$}\NormalTok{Rivers.Impaired.mi2 <-}\StringTok{ }\KeywordTok{as.numeric}\NormalTok{(Rivers}\OperatorTok{$}\NormalTok{Rivers.Impaired.mi2)}
\NormalTok{Rivers}\OperatorTok{$}\NormalTok{Rivers.Impaired.percent <-}\StringTok{ }\KeywordTok{as.numeric}\NormalTok{(Rivers}\OperatorTok{$}\NormalTok{Rivers.Impaired.percent)}
\NormalTok{Rivers}\OperatorTok{$}\NormalTok{Rivers.Impaired.percent.TMDL <-}\StringTok{ }\KeywordTok{as.numeric}\NormalTok{(Rivers}\OperatorTok{$}\NormalTok{Rivers.Impaired.percent.TMDL)}
\KeywordTok{str}\NormalTok{(Rivers)}
\end{Highlighting}
\end{Shaded}

\begin{verbatim}
## 'data.frame':    50 obs. of  6 variables:
##  $ State                       : Factor w/ 50 levels "Alabama","Alaska",..: 1 2 3 4 5 6 7 8 9 10 ...
##  $ Rivers.Assessed.mi2         : num  10538 602 2764 9979 32803 ...
##  $ Rivers.Assessed.percent     : num  14 0 3 11 16 56 41 100 20 19 ...
##  $ Rivers.Impaired.mi2         : num  1146 15 144 1440 13350 ...
##  $ Rivers.Impaired.percent     : num  11 2 5 14 41 0 0 88 53 9 ...
##  $ Rivers.Impaired.percent.TMDL: num  53 100 6 2 NA 14 73 37 NA 78 ...
\end{verbatim}

\begin{enumerate}
\def\labelenumi{\arabic{enumi}.}
\setcounter{enumi}{5}
\tightlist
\item
  Scrape the Lakes table, with every column except year. Then, turn it
  into a data frame.
\end{enumerate}

\begin{Shaded}
\begin{Highlighting}[]
\NormalTok{State <-}\StringTok{ }\NormalTok{webpage }\OperatorTok\StringTok{ }\KeywordTok{html_nodes}\NormalTok{(}\StringTok{"table:nth-child(14) td:nth-child(1)"}\NormalTok{) }\OperatorTok\StringTok{ }\KeywordTok{html_text}\NormalTok{()}
\NormalTok{Lakes.Assessed.acres <-}\StringTok{ }\NormalTok{webpage }\OperatorTok\StringTok{ }\KeywordTok{html_nodes}\NormalTok{(}\StringTok{"table:nth-child(14) td:nth-child(2)"}\NormalTok{) }\OperatorTok\StringTok{ }\KeywordTok{html_text}\NormalTok{()}
\NormalTok{Lakes.Assessed.percent <-}\StringTok{ }\NormalTok{webpage }\OperatorTok\StringTok{ }\KeywordTok{html_nodes}\NormalTok{(}\StringTok{"table:nth-child(14) td:nth-child(3)"}\NormalTok{) }\OperatorTok\StringTok{ }\KeywordTok{html_text}\NormalTok{()}
\NormalTok{Lakes.Impaired.acres <-}\StringTok{ }\NormalTok{webpage }\OperatorTok\StringTok{ }\KeywordTok{html_nodes}\NormalTok{(}\StringTok{"table:nth-child(14) td:nth-child(4)"}\NormalTok{) }\OperatorTok\StringTok{ }\KeywordTok{html_text}\NormalTok{()}
\NormalTok{Lakes.Impaired.percent <-}\StringTok{ }\NormalTok{webpage }\OperatorTok\StringTok{ }\KeywordTok{html_nodes}\NormalTok{(}\StringTok{"table:nth-child(14) td:nth-child(5)"}\NormalTok{) }\OperatorTok\StringTok{ }\KeywordTok{html_text}\NormalTok{()}
\NormalTok{Lakes.Impaired.percent.TMDL <-}\StringTok{ }\NormalTok{webpage }\OperatorTok\StringTok{ }\KeywordTok{html_nodes}\NormalTok{(}\StringTok{"table:nth-child(14) td:nth-child(6)"}\NormalTok{) }\OperatorTok\StringTok{ }\KeywordTok{html_text}\NormalTok{()}

\NormalTok{Lakes <-}\StringTok{ }\KeywordTok{data.frame}\NormalTok{(State, Lakes.Assessed.acres, Lakes.Assessed.percent, }
\NormalTok{                    Lakes.Impaired.acres, Lakes.Impaired.percent, }
\NormalTok{                    Lakes.Impaired.percent.TMDL)}
\end{Highlighting}
\end{Shaded}

\begin{enumerate}
\def\labelenumi{\arabic{enumi}.}
\setcounter{enumi}{6}
\item
  Filter out the states with no data.
\item
  Use \texttt{str\_replace} to remove non-numeric characters from the
  numeric columns.
\item
  Set the numeric columns to a numeric class and verify this using
  \texttt{str}.
\end{enumerate}

\begin{Shaded}
\begin{Highlighting}[]
\CommentTok{# 7 Filter out States with no data}
\NormalTok{Lakes <-}\StringTok{ }\NormalTok{Lakes }\OperatorTok
\StringTok{  }\KeywordTok{filter}\NormalTok{(State }\OperatorTok{!=}\StringTok{ "Hawaii"} \OperatorTok{&}\StringTok{ }\NormalTok{State }\OperatorTok{!=}\StringTok{ "Pennsylvania"}\NormalTok{)}

\CommentTok{# 8 Remove non-numeric characters}
\NormalTok{Lakes}\OperatorTok{$}\NormalTok{Lakes.Assessed.acres <-}\StringTok{ }\KeywordTok{str_replace}\NormalTok{(Lakes}\OperatorTok{$}\NormalTok{Lakes.Assessed.acres,}
                                          \DataTypeTok{pattern =} \StringTok{"([,])"}\NormalTok{, }\DataTypeTok{replacement =} \StringTok{""}\NormalTok{)}
\NormalTok{Lakes}\OperatorTok{$}\NormalTok{Lakes.Assessed.percent <-}\KeywordTok{str_replace}\NormalTok{(Lakes}\OperatorTok{$}\NormalTok{Lakes.Assessed.percent,}
                                           \DataTypeTok{pattern =} \StringTok{"([%])"}\NormalTok{, }\DataTypeTok{replacement =} \StringTok{""}\NormalTok{)}
\NormalTok{Lakes}\OperatorTok{$}\NormalTok{Lakes.Assessed.percent <-}\KeywordTok{str_replace}\NormalTok{(Lakes}\OperatorTok{$}\NormalTok{Lakes.Assessed.percent,}
                                           \DataTypeTok{pattern =} \StringTok{"([*])"}\NormalTok{, }\DataTypeTok{replacement =} \StringTok{""}\NormalTok{)}
\NormalTok{Lakes}\OperatorTok{$}\NormalTok{Lakes.Impaired.acres <-}\StringTok{ }\KeywordTok{str_replace}\NormalTok{(Lakes}\OperatorTok{$}\NormalTok{Lakes.Impaired.acres,}
                                          \DataTypeTok{pattern =} \StringTok{"([,])"}\NormalTok{, }\DataTypeTok{replacement =} \StringTok{""}\NormalTok{)}
\NormalTok{Lakes}\OperatorTok{$}\NormalTok{Lakes.Impaired.percent <-}\StringTok{ }\KeywordTok{str_replace}\NormalTok{(Lakes}\OperatorTok{$}\NormalTok{Lakes.Impaired.percent,}
                                            \DataTypeTok{pattern =} \StringTok{"([%])"}\NormalTok{, }\DataTypeTok{replacement =} \StringTok{""}\NormalTok{)}
\NormalTok{Lakes}\OperatorTok{$}\NormalTok{Lakes.Impaired.percent.TMDL <-}\StringTok{ }\KeywordTok{str_replace}\NormalTok{(Lakes}\OperatorTok{$}\NormalTok{Lakes.Impaired.percent.TMDL,}
                                                 \DataTypeTok{pattern =} \StringTok{"([±])"}\NormalTok{, }\DataTypeTok{replacement =} \StringTok{""}\NormalTok{)}
\NormalTok{Lakes}\OperatorTok{$}\NormalTok{Lakes.Impaired.percent.TMDL <-}\StringTok{ }\KeywordTok{str_replace}\NormalTok{(Lakes}\OperatorTok{$}\NormalTok{Lakes.Impaired.percent.TMDL,}
                                                 \DataTypeTok{pattern =} \StringTok{"([%])"}\NormalTok{, }\DataTypeTok{replacement =} \StringTok{""}\NormalTok{)}

\CommentTok{# 9 Set as numeric}
\NormalTok{Lakes}\OperatorTok{$}\NormalTok{Lakes.Assessed.acres <-}\StringTok{ }\KeywordTok{as.numeric}\NormalTok{(Lakes}\OperatorTok{$}\NormalTok{Lakes.Assessed.acres)}
\end{Highlighting}
\end{Shaded}

\begin{verbatim}
## Warning: NAs introduced by coercion
\end{verbatim}

\begin{Shaded}
\begin{Highlighting}[]
\NormalTok{Lakes}\OperatorTok{$}\NormalTok{Lakes.Assessed.percent <-}\StringTok{ }\KeywordTok{as.numeric}\NormalTok{(Lakes}\OperatorTok{$}\NormalTok{Lakes.Assessed.percent)}
\NormalTok{Lakes}\OperatorTok{$}\NormalTok{Lakes.Impaired.acres <-}\StringTok{ }\KeywordTok{as.numeric}\NormalTok{(Lakes}\OperatorTok{$}\NormalTok{Lakes.Impaired.acres)}
\NormalTok{Lakes}\OperatorTok{$}\NormalTok{Lakes.Impaired.percent <-}\StringTok{ }\KeywordTok{as.numeric}\NormalTok{(Lakes}\OperatorTok{$}\NormalTok{Lakes.Impaired.percent)}
\NormalTok{Lakes}\OperatorTok{$}\NormalTok{Lakes.Impaired.percent.TMDL <-}\StringTok{ }\KeywordTok{as.numeric}\NormalTok{(Lakes}\OperatorTok{$}\NormalTok{Lakes.Impaired.percent.TMDL)}
\KeywordTok{str}\NormalTok{(Lakes)}
\end{Highlighting}
\end{Shaded}

\begin{verbatim}
## 'data.frame':    48 obs. of  6 variables:
##  $ State                      : Factor w/ 50 levels "Alabama","Alaska",..: 1 2 3 4 5 6 7 8 9 10 ...
##  $ Lakes.Assessed.acres       : num  431 5981 114976 64778 NA ...
##  $ Lakes.Assessed.percent     : num  88 0 34 13 50 95 47 100 54 82 ...
##  $ Lakes.Impaired.acres       : num  81740 1137 4895 6513 473954 ...
##  $ Lakes.Impaired.percent     : num  19 19 4 10 45 7 12 88 82 2 ...
##  $ Lakes.Impaired.percent.TMDL: num  53 73 9 71 NA 0 7 69 NA 20 ...
\end{verbatim}

\begin{enumerate}
\def\labelenumi{\arabic{enumi}.}
\setcounter{enumi}{9}
\tightlist
\item
  Join the two data frames with a \texttt{full\_join}.
\end{enumerate}

\begin{Shaded}
\begin{Highlighting}[]
\CommentTok{#Full Join table of river and lakes}
\NormalTok{Rivers_lakes_combined <-}\StringTok{ }\KeywordTok{full_join}\NormalTok{(Rivers, Lakes)}
\end{Highlighting}
\end{Shaded}

\begin{verbatim}
## Joining, by = "State"
\end{verbatim}

\begin{enumerate}
\def\labelenumi{\arabic{enumi}.}
\setcounter{enumi}{10}
\tightlist
\item
  Create one graph that compares the data for lakes and/or rivers. This
  option is flexible; choose a relationship (or relationships) that seem
  interesting to you, and think about the implications of your findings.
  This graph should be edited so it follows best data visualization
  practices.
\end{enumerate}

(You may choose to run a statistical test or add a line of best fit;
this is optional but may aid in your interpretations)

\begin{Shaded}
\begin{Highlighting}[]
\KeywordTok{ggplot}\NormalTok{(Rivers_lakes_combined, }\KeywordTok{aes}\NormalTok{(}\DataTypeTok{x =}\NormalTok{ Lakes.Assessed.acres, }\DataTypeTok{y =}\NormalTok{ Lakes.Impaired.acres, }
                                  \DataTypeTok{fill =}\NormalTok{ Lakes.Impaired.percent.TMDL)) }\OperatorTok{+}
\StringTok{  }\KeywordTok{geom_point}\NormalTok{(}\DataTypeTok{shape =} \DecValTok{21}\NormalTok{, }\DataTypeTok{size =} \DecValTok{2}\NormalTok{, }\DataTypeTok{alpha =} \FloatTok{0.7}\NormalTok{) }\OperatorTok{+}
\StringTok{  }\KeywordTok{scale_fill_viridis}\NormalTok{(}\DataTypeTok{option =} \StringTok{"magma"}\NormalTok{, }\DataTypeTok{begin =} \FloatTok{0.2}\NormalTok{, }\DataTypeTok{end =} \FloatTok{0.9}\NormalTok{, }\DataTypeTok{direction =} \DecValTok{-1}\NormalTok{) }\OperatorTok{+}
\StringTok{  }\KeywordTok{ylim}\NormalTok{(}\DecValTok{0}\NormalTok{, }\DecValTok{50000}\NormalTok{) }\OperatorTok{+}
\StringTok{  }\KeywordTok{geom_label_repel}\NormalTok{(}\KeywordTok{aes}\NormalTok{(}\DataTypeTok{label =}\NormalTok{ State), }\DataTypeTok{nudge_x =} \DecValTok{-4}\NormalTok{, }\DataTypeTok{nudge_y =} \DecValTok{-4}\NormalTok{, }
                   \DataTypeTok{size =} \DecValTok{3}\NormalTok{, }\DataTypeTok{alpha =} \FloatTok{0.8}\NormalTok{) }\OperatorTok{+}
\StringTok{  }\KeywordTok{labs}\NormalTok{(}\DataTypeTok{x =} \StringTok{"Lakes Assessed (acres)"}\NormalTok{,}
        \DataTypeTok{y =} \StringTok{"Lakes Impaired (acres)"}\NormalTok{, }
        \DataTypeTok{fill =} \StringTok{"% Impaired with TMDL"}\NormalTok{)}
\end{Highlighting}
\end{Shaded}

\begin{verbatim}
## Warning: Removed 24 rows containing missing values (geom_point).
\end{verbatim}

\begin{verbatim}
## Warning: Removed 24 rows containing missing values (geom_label_repel).
\end{verbatim}

\includegraphics{A10_DataScraping_files/figure-latex/unnamed-chunk-8-1.pdf}

\begin{enumerate}
\def\labelenumi{\arabic{enumi}.}
\setcounter{enumi}{11}
\tightlist
\item
  Summarize the findings that accompany your graph. You may choose to
  suggest further research or data collection to help explain the
  results.
\end{enumerate}

\begin{quote}
Lakes in states that have low assessed acres also have low assessed
impaired acres. However, these lakes have a higher percent impaired with
TMDL. These findings suggest that more lakes need to be assessed in
order to determine if the lake is impaired, and those will affect the
percent impaired with TMDL too. In addition, the data holds a lot of NAs
in percent impaired with TMDL so data collection is needed for the Lakes
dataset to be accurate.
\end{quote}


\end{document}
