\documentclass[]{article}
\usepackage{lmodern}
\usepackage{amssymb,amsmath}
\usepackage{ifxetex,ifluatex}
\usepackage{fixltx2e} % provides \textsubscript
\ifnum 0\ifxetex 1\fi\ifluatex 1\fi=0 % if pdftex
  \usepackage[T1]{fontenc}
  \usepackage[utf8]{inputenc}
\else % if luatex or xelatex
  \ifxetex
    \usepackage{mathspec}
  \else
    \usepackage{fontspec}
  \fi
  \defaultfontfeatures{Ligatures=TeX,Scale=MatchLowercase}
\fi
% use upquote if available, for straight quotes in verbatim environments
\IfFileExists{upquote.sty}{\usepackage{upquote}}{}
% use microtype if available
\IfFileExists{microtype.sty}{%
\usepackage{microtype}
\UseMicrotypeSet[protrusion]{basicmath} % disable protrusion for tt fonts
}{}
\usepackage[margin=2.54cm]{geometry}
\usepackage{hyperref}
\hypersetup{unicode=true,
            pdftitle={Assignment 3: Data Exploration},
            pdfauthor={Nikki Shintaku},
            pdfborder={0 0 0},
            breaklinks=true}
\urlstyle{same}  % don't use monospace font for urls
\usepackage{color}
\usepackage{fancyvrb}
\newcommand{\VerbBar}{|}
\newcommand{\VERB}{\Verb[commandchars=\\\{\}]}
\DefineVerbatimEnvironment{Highlighting}{Verbatim}{commandchars=\\\{\}}
% Add ',fontsize=\small' for more characters per line
\usepackage{framed}
\definecolor{shadecolor}{RGB}{248,248,248}
\newenvironment{Shaded}{\begin{snugshade}}{\end{snugshade}}
\newcommand{\AlertTok}[1]{\textcolor[rgb]{0.94,0.16,0.16}{#1}}
\newcommand{\AnnotationTok}[1]{\textcolor[rgb]{0.56,0.35,0.01}{\textbf{\textit{#1}}}}
\newcommand{\AttributeTok}[1]{\textcolor[rgb]{0.77,0.63,0.00}{#1}}
\newcommand{\BaseNTok}[1]{\textcolor[rgb]{0.00,0.00,0.81}{#1}}
\newcommand{\BuiltInTok}[1]{#1}
\newcommand{\CharTok}[1]{\textcolor[rgb]{0.31,0.60,0.02}{#1}}
\newcommand{\CommentTok}[1]{\textcolor[rgb]{0.56,0.35,0.01}{\textit{#1}}}
\newcommand{\CommentVarTok}[1]{\textcolor[rgb]{0.56,0.35,0.01}{\textbf{\textit{#1}}}}
\newcommand{\ConstantTok}[1]{\textcolor[rgb]{0.00,0.00,0.00}{#1}}
\newcommand{\ControlFlowTok}[1]{\textcolor[rgb]{0.13,0.29,0.53}{\textbf{#1}}}
\newcommand{\DataTypeTok}[1]{\textcolor[rgb]{0.13,0.29,0.53}{#1}}
\newcommand{\DecValTok}[1]{\textcolor[rgb]{0.00,0.00,0.81}{#1}}
\newcommand{\DocumentationTok}[1]{\textcolor[rgb]{0.56,0.35,0.01}{\textbf{\textit{#1}}}}
\newcommand{\ErrorTok}[1]{\textcolor[rgb]{0.64,0.00,0.00}{\textbf{#1}}}
\newcommand{\ExtensionTok}[1]{#1}
\newcommand{\FloatTok}[1]{\textcolor[rgb]{0.00,0.00,0.81}{#1}}
\newcommand{\FunctionTok}[1]{\textcolor[rgb]{0.00,0.00,0.00}{#1}}
\newcommand{\ImportTok}[1]{#1}
\newcommand{\InformationTok}[1]{\textcolor[rgb]{0.56,0.35,0.01}{\textbf{\textit{#1}}}}
\newcommand{\KeywordTok}[1]{\textcolor[rgb]{0.13,0.29,0.53}{\textbf{#1}}}
\newcommand{\NormalTok}[1]{#1}
\newcommand{\OperatorTok}[1]{\textcolor[rgb]{0.81,0.36,0.00}{\textbf{#1}}}
\newcommand{\OtherTok}[1]{\textcolor[rgb]{0.56,0.35,0.01}{#1}}
\newcommand{\PreprocessorTok}[1]{\textcolor[rgb]{0.56,0.35,0.01}{\textit{#1}}}
\newcommand{\RegionMarkerTok}[1]{#1}
\newcommand{\SpecialCharTok}[1]{\textcolor[rgb]{0.00,0.00,0.00}{#1}}
\newcommand{\SpecialStringTok}[1]{\textcolor[rgb]{0.31,0.60,0.02}{#1}}
\newcommand{\StringTok}[1]{\textcolor[rgb]{0.31,0.60,0.02}{#1}}
\newcommand{\VariableTok}[1]{\textcolor[rgb]{0.00,0.00,0.00}{#1}}
\newcommand{\VerbatimStringTok}[1]{\textcolor[rgb]{0.31,0.60,0.02}{#1}}
\newcommand{\WarningTok}[1]{\textcolor[rgb]{0.56,0.35,0.01}{\textbf{\textit{#1}}}}
\usepackage{graphicx,grffile}
\makeatletter
\def\maxwidth{\ifdim\Gin@nat@width>\linewidth\linewidth\else\Gin@nat@width\fi}
\def\maxheight{\ifdim\Gin@nat@height>\textheight\textheight\else\Gin@nat@height\fi}
\makeatother
% Scale images if necessary, so that they will not overflow the page
% margins by default, and it is still possible to overwrite the defaults
% using explicit options in \includegraphics[width, height, ...]{}
\setkeys{Gin}{width=\maxwidth,height=\maxheight,keepaspectratio}
\IfFileExists{parskip.sty}{%
\usepackage{parskip}
}{% else
\setlength{\parindent}{0pt}
\setlength{\parskip}{6pt plus 2pt minus 1pt}
}
\setlength{\emergencystretch}{3em}  % prevent overfull lines
\providecommand{\tightlist}{%
  \setlength{\itemsep}{0pt}\setlength{\parskip}{0pt}}
\setcounter{secnumdepth}{0}
% Redefines (sub)paragraphs to behave more like sections
\ifx\paragraph\undefined\else
\let\oldparagraph\paragraph
\renewcommand{\paragraph}[1]{\oldparagraph{#1}\mbox{}}
\fi
\ifx\subparagraph\undefined\else
\let\oldsubparagraph\subparagraph
\renewcommand{\subparagraph}[1]{\oldsubparagraph{#1}\mbox{}}
\fi

%%% Use protect on footnotes to avoid problems with footnotes in titles
\let\rmarkdownfootnote\footnote%
\def\footnote{\protect\rmarkdownfootnote}

%%% Change title format to be more compact
\usepackage{titling}

% Create subtitle command for use in maketitle
\providecommand{\subtitle}[1]{
  \posttitle{
    \begin{center}\large#1\end{center}
    }
}

\setlength{\droptitle}{-2em}

  \title{Assignment 3: Data Exploration}
    \pretitle{\vspace{\droptitle}\centering\huge}
  \posttitle{\par}
    \author{Nikki Shintaku}
    \preauthor{\centering\large\emph}
  \postauthor{\par}
    \date{}
    \predate{}\postdate{}
  

\begin{document}
\maketitle

\hypertarget{overview}{%
\subsection{OVERVIEW}\label{overview}}

This exercise accompanies the lessons in Environmental Data Analytics on
Data Exploration.

\hypertarget{directions}{%
\subsection{Directions}\label{directions}}

\begin{enumerate}
\def\labelenumi{\arabic{enumi}.}
\tightlist
\item
  Change ``Student Name'' on line 3 (above) with your name.
\item
  Work through the steps, \textbf{creating code and output} that fulfill
  each instruction.
\item
  Be sure to \textbf{answer the questions} in this assignment document.
\item
  When you have completed the assignment, \textbf{Knit} the text and
  code into a single PDF file.
\item
  After Knitting, submit the completed exercise (PDF file) to the
  dropbox in Sakai. Add your last name into the file name (e.g.,
  ``Salk\_A03\_DataExploration.Rmd'') prior to submission.
\end{enumerate}

The completed exercise is due on Tuesday, January 28 at 1:00 pm.

\hypertarget{set-up-your-r-session}{%
\subsection{Set up your R session}\label{set-up-your-r-session}}

\begin{enumerate}
\def\labelenumi{\arabic{enumi}.}
\tightlist
\item
  Check your working directory, load necessary packages (tidyverse), and
  upload two datasets: the ECOTOX neonicotinoid dataset
  (ECOTOX\_Neonicotinoids\_Insects\_raw.csv) and the Niwot Ridge NEON
  dataset for litter and woody debris
  (NEON\_NIWO\_Litter\_massdata\_2018-08\_raw.csv). Name these datasets
  ``Neonics'' and ``Litter'', respectively.
\end{enumerate}

\begin{Shaded}
\begin{Highlighting}[]
\KeywordTok{getwd}\NormalTok{()}
\end{Highlighting}
\end{Shaded}

\begin{verbatim}
## [1] "/Users/nikkishintaku/Desktop/Environmental872/Environmental_Data_Analytics_2020"
\end{verbatim}

\begin{Shaded}
\begin{Highlighting}[]
\KeywordTok{library}\NormalTok{(knitr)}
\NormalTok{opts_knit}\OperatorTok{$}\KeywordTok{set}\NormalTok{(}\DataTypeTok{root.dir =} \StringTok{"/Users/nikkishintaku/Desktop/Environmental872/Environmental_Data_Analytics_2020"}\NormalTok{)}
\end{Highlighting}
\end{Shaded}

\begin{Shaded}
\begin{Highlighting}[]
\KeywordTok{library}\NormalTok{(tidyverse)}

\CommentTok{#reading in Neonics and Litter datasets }
\NormalTok{Neonics <-}\StringTok{ }\KeywordTok{read.csv}\NormalTok{(}\StringTok{"./Data/Raw/ECOTOX_Neonicotinoids_Insects_raw.csv"}\NormalTok{)}
\NormalTok{Litter <-}\StringTok{ }\KeywordTok{read.csv}\NormalTok{(}\StringTok{"./Data/Raw/NEON_NIWO_Litter_massdata_2018-08_raw.csv"}\NormalTok{)}
\end{Highlighting}
\end{Shaded}

\hypertarget{learn-about-your-system}{%
\subsection{Learn about your system}\label{learn-about-your-system}}

\begin{enumerate}
\def\labelenumi{\arabic{enumi}.}
\setcounter{enumi}{1}
\tightlist
\item
  The neonicotinoid dataset was collected from the Environmental
  Protection Agency's ECOTOX Knowledgebase, a database for ecotoxicology
  research. Neonicotinoids are a class of insecticides used widely in
  agriculture. The dataset that has been pulled includes all studies
  published on insects. Why might we be interested in the ecotoxicology
  of neonicotinoids on insects? Feel free to do a brief internet search
  if you feel you need more background information.
\end{enumerate}

\begin{quote}
Answer: If companies are using neonicotinoids as insecticides on
agriculture, then we need to know the ecotoxicology of them in order to
assess if the neonicotinoids are going to be harmful to us when we eat
the crops they are put on. We also would need to make sure that it is
targeting the correct insect and not harming other animals.
\end{quote}

\begin{enumerate}
\def\labelenumi{\arabic{enumi}.}
\setcounter{enumi}{2}
\tightlist
\item
  The Niwot Ridge litter and woody debris dataset was collected from the
  National Ecological Observatory Network, which collectively includes
  81 aquatic and terrestrial sites across 20 ecoclimatic domains. 32 of
  these sites sample forest litter and woody debris, and we will focus
  on the Niwot Ridge long-term ecological research (LTER) station in
  Colorado. Why might we be interested in studying litter and woody
  debris that falls to the ground in forests? Feel free to do a brief
  internet search if you feel you need more background information.
\end{enumerate}

\begin{quote}
Answer: Studying litter and woody debris that falls to the ground is
important data that can be used to estimate annual aboveground net
primary productivity and aboveground biomass in the region. It can also
help to understand vegetative carbon fluxes over time in the particular
ecoclimate domain. Litter and woody debris hold an important role in
carbon budgets and nutrient cycling.
\end{quote}

\begin{enumerate}
\def\labelenumi{\arabic{enumi}.}
\setcounter{enumi}{3}
\tightlist
\item
  How is litter and woody debris sampled as part of the NEON network?
  Read the NEON\_Litterfall\_UserGuide.pdf document to learn more. List
  three pieces of salient information about the sampling methods here:
\end{enumerate}

\begin{quote}
Answer: \emph{Sampling locations were selected randomly and sampling
occurred in tower plots. In sites with forested tower airsheds, there
was 20 40mX40m plots. In sites with low-statured vegetation, there was 4
40mx40m plots and 26 20mx20m plots.\\
}There was one elevated and one ground trap was deployed for every
400m\^{}2 plot area. Trap placements within plots were targeted or
randomized, depending on vegetation *Ground traps were sampled once per
year. Elevated traps sampling frequency varied by vegetation present at
the site with 1x every 2 weeks in deciduous forest and 1x every 1-2
months in evergreen sites.
\end{quote}

\hypertarget{obtain-basic-summaries-of-your-data-neonics}{%
\subsection{Obtain basic summaries of your data
(Neonics)}\label{obtain-basic-summaries-of-your-data-neonics}}

\begin{enumerate}
\def\labelenumi{\arabic{enumi}.}
\setcounter{enumi}{4}
\tightlist
\item
  What are the dimensions of the dataset?
\end{enumerate}

\begin{Shaded}
\begin{Highlighting}[]
\CommentTok{#dimensions of Neonics}
\KeywordTok{dim}\NormalTok{(Neonics)}
\end{Highlighting}
\end{Shaded}

\begin{verbatim}
## [1] 4623   30
\end{verbatim}

\begin{enumerate}
\def\labelenumi{\arabic{enumi}.}
\setcounter{enumi}{5}
\tightlist
\item
  Using the \texttt{summary} function, determine the most common effects
  that are studied. Why might these effects specifically be of interest?
\end{enumerate}

\begin{Shaded}
\begin{Highlighting}[]
\CommentTok{#summary of only the Effects column of the Neonics dataset}
\KeywordTok{summary}\NormalTok{(Neonics}\OperatorTok{$}\NormalTok{Effect) }\CommentTok{#effect group }
\end{Highlighting}
\end{Shaded}

\begin{verbatim}
##     Accumulation        Avoidance         Behavior     Biochemistry 
##               12              102              360               11 
##          Cell(s)      Development        Enzyme(s) Feeding behavior 
##                9              136               62              255 
##         Genetics           Growth        Histology       Hormone(s) 
##               82               38                5                1 
##    Immunological     Intoxication       Morphology        Mortality 
##               16               12               22             1493 
##       Physiology       Population     Reproduction 
##                7             1803              197
\end{verbatim}

\begin{Shaded}
\begin{Highlighting}[]
\KeywordTok{summary}\NormalTok{(Neonics}\OperatorTok{$}\NormalTok{Effect.Measurement) }\CommentTok{#effect and measurement}
\end{Highlighting}
\end{Shaded}

\begin{verbatim}
##                                           Abundance 
##                                                1699 
##                                           Mortality 
##                                                1294 
##                                            Survival 
##                                                 133 
##                              Progeny counts/numbers 
##                                                 120 
##                                    Food consumption 
##                                                 103 
##                                           Emergence 
##                                                  98 
##                      Search/explore/forage behavior 
##                                                  96 
##                           Feeding behavior, general 
##                                                  92 
##                                  Chemical avoidance 
##                                                  65 
##                                              Weight 
##                                                  48 
##           Distance moved, change in direct movement 
##                                                  38 
##                                    Feeding behavior 
##                                                  36 
##                                     Flying behavior 
##                                                  30 
##               Accuracy of learned task, performance 
##                                                  28 
##                                           Sex ratio 
##                                                  27 
##                                           Fecundity 
##                                                  26 
##                                  Stimulus avoidance 
##                                                  26 
##                                   Righting response 
##                                                  24 
##                                            Lifespan 
##                                                  23 
##                                       Acquired task 
##                                                  22 
##                                               Hatch 
##                                                  21 
##                                  Predatory behavior 
##                                                  21 
##                                Acetylcholinesterase 
##                                                  20 
##                                                Walk 
##                                                  19 
##                                   Freezing behavior 
##                                                  18 
##                      Reproductive success (general) 
##                                                  17 
##        Slowed, Retarded, Delayed or Non-development 
##                                                  17 
##                                            Grooming 
##                                                  16 
##                                            Diameter 
##                                                  14 
##                                             Residue 
##                                                  12 
##                                   Activity, general 
##                                                  11 
##                                      Food avoidance 
##                                                  11 
##                                             Control 
##                                                   9 
##                      Developmental changes, general 
##                                                   9 
##                          Intrinsic rate of increase 
##                                                   9 
##                                    Pollen collected 
##                                                   9 
##                                                Size 
##                                                   9 
##                                            Esterase 
##                                                   8 
##                               Intoxication, general 
##                                                   8 
##                         Mortality/survival, general 
##                                                   8 
##      Population change (change in N/change in time) 
##                                                   8 
##                                         Smell/Sniff 
##                                                   8 
##                                             Biomass 
##                                                   7 
##                                       Catalase mRNA 
##                                                   7 
##                                     Generation time 
##                                                   7 
##                                            Infected 
##                                                   7 
##                                         Orientation 
##                                                   7 
##                            Population doubling time 
##                                                   7 
##                              Population growth rate 
##                                                   7 
##                                        Sealed brood 
##                                                   7 
##                                   Vitellogenin mRNA 
##                                                   7 
##                                        Ali esterase 
##                                                   6 
## Apoptosis, programmed cell death, DNA fragmentation 
##                                                   6 
##                                    Carboxylesterase 
##                                                   6 
##                                            Hemocyte 
##                                                   6 
##                                           Knockdown 
##                                                   6 
##                                           Viability 
##                                                   6 
##                                          Extinction 
##                                                   5 
##                               Net Reproductive Rate 
##                                                   5 
##                                  Polyphenol oxidase 
##                                                   5 
##                                    Prey penetration 
##                                                   5 
##                                            Pupation 
##                                                   5 
##                               Reproducing organisms 
##                                                   5 
##   Amount or percent animals infested with parasites 
##                                                   4 
##              Continual reinforcement task performed 
##                                                   4 
##                                     Defensin 1 mRNA 
##                                                   4 
##                                 Diversity, Evenness 
##                                                   4 
##              Encapsulation or Melanization Response 
##                                                   4 
##                          General biochemical effect 
##                                                   4 
##                           Glutathione S-transferase 
##                                                   4 
##                       Histological changes, general 
##                                                   4 
##                                     Life expectancy 
##                                                   4 
##                         Thioredoxin peroxidase mRNA 
##                                                   4 
##                      Vanin-like protein 1-like mRNA 
##                                                   4 
##                                   Bees wax produced 
##                                                   3 
##                         Behavioral changes, general 
##                                                   3 
##                                            Catalase 
##                                                   3 
##                                       Cell turnover 
##                                                   3 
##                                    Cytochrome P-450 
##                                                   3 
##                                        Feeding time 
##                                                   3 
##                                              Length 
##                                                   3 
##                                      Protein, total 
##                                                   3 
##                                         Respiration 
##                                                   3 
##                         Response time to a stimulus 
##                                                   3 
##                                               Stage 
##                                                   3 
##                               Time to first progeny 
##                                                   3 
##                                      Trehalase mRNA 
##                                                   3 
##                                Alkaline phosphatase 
##                                                   2 
##             Carboxylesterase clade I, member 1 mRNA 
##                                                   2 
##                                     Centractin mRNA 
##                                                   2 
##                                    Chitinase 5 mRNA 
##                                                   2 
##                           Colony maintenance (bees) 
##                                                   2 
##                                           COX2 mRNA 
##                                                   2 
##                               Endoplasmin-like mRNA 
##                                                   2 
##                                   Gamete production 
##                                                   2 
##                        Glucose dehydrogenase 2 mRNA 
##                                                   2 
##                       Glucosinolate sulphatase mRNA 
##                                                   2 
##                  Glutathione peroxidase-like 1 mRNA 
##                                                   2 
##                  Glutathione peroxidase-like 2 mRNA 
##                                                   2 
##                                             (Other) 
##                                                  77
\end{verbatim}

\begin{quote}
Answer: The most common effect is population and mortality with the most
common effect and measurement being abundance and mortality. Using the
summary function is an easy way to quickly see how many cases of each
effect were found in all the studies about the effects of neonicotinoids
on insects. This gives a clear idea of more specific effects the
neonicotinoids have on insects and can be used to steer further research
and understanding of how it can help or hurt agriculture growth.
\end{quote}

\begin{enumerate}
\def\labelenumi{\arabic{enumi}.}
\setcounter{enumi}{6}
\tightlist
\item
  Using the \texttt{summary} function, determine the six most commonly
  studied species in the dataset (common name). What do these species
  have in common, and why might they be of interest over other insects?
  Feel free to do a brief internet search for more information if
  needed.
\end{enumerate}

\begin{Shaded}
\begin{Highlighting}[]
\KeywordTok{summary}\NormalTok{(Neonics}\OperatorTok{$}\NormalTok{Species.Common.Name)}
\end{Highlighting}
\end{Shaded}

\begin{verbatim}
##                          Honey Bee                     Parasitic Wasp 
##                                667                                285 
##              Buff Tailed Bumblebee                Carniolan Honey Bee 
##                                183                                152 
##                         Bumble Bee                   Italian Honeybee 
##                                140                                113 
##                    Japanese Beetle                  Asian Lady Beetle 
##                                 94                                 76 
##                     Euonymus Scale                           Wireworm 
##                                 75                                 69 
##                  European Dark Bee                  Minute Pirate Bug 
##                                 66                                 62 
##               Asian Citrus Psyllid                      Parastic Wasp 
##                                 60                                 58 
##             Colorado Potato Beetle                    Parasitoid Wasp 
##                                 57                                 51 
##                Erythrina Gall Wasp                       Beetle Order 
##                                 49                                 47 
##        Snout Beetle Family, Weevil           Sevenspotted Lady Beetle 
##                                 47                                 46 
##                     True Bug Order              Buff-tailed Bumblebee 
##                                 45                                 39 
##                       Aphid Family                     Cabbage Looper 
##                                 38                                 38 
##               Sweetpotato Whitefly                      Braconid Wasp 
##                                 37                                 33 
##                       Cotton Aphid                     Predatory Mite 
##                                 33                                 33 
##             Ladybird Beetle Family                         Parasitoid 
##                                 30                                 30 
##                      Scarab Beetle                      Spring Tiphia 
##                                 29                                 29 
##                        Thrip Order               Ground Beetle Family 
##                                 29                                 27 
##                 Rove Beetle Family                      Tobacco Aphid 
##                                 27                                 27 
##                       Chalcid Wasp             Convergent Lady Beetle 
##                                 25                                 25 
##                      Stingless Bee                  Spider/Mite Class 
##                                 25                                 24 
##                Tobacco Flea Beetle                   Citrus Leafminer 
##                                 24                                 23 
##                    Ladybird Beetle                          Mason Bee 
##                                 23                                 22 
##                           Mosquito                      Argentine Ant 
##                                 22                                 21 
##                             Beetle         Flatheaded Appletree Borer 
##                                 21                                 20 
##               Horned Oak Gall Wasp                 Leaf Beetle Family 
##                                 20                                 20 
##                  Potato Leafhopper         Tooth-necked Fungus Beetle 
##                                 20                                 20 
##                       Codling Moth          Black-spotted Lady Beetle 
##                                 19                                 18 
##                       Calico Scale                Fairyfly Parasitoid 
##                                 18                                 18 
##                        Lady Beetle             Minute Parasitic Wasps 
##                                 18                                 18 
##                          Mirid Bug                   Mulberry Pyralid 
##                                 18                                 18 
##                           Silkworm                     Vedalia Beetle 
##                                 18                                 18 
##              Araneoid Spider Order                          Bee Order 
##                                 17                                 17 
##                     Egg Parasitoid                       Insect Class 
##                                 17                                 17 
##           Moth And Butterfly Order       Oystershell Scale Parasitoid 
##                                 17                                 17 
## Hemlock Woolly Adelgid Lady Beetle              Hemlock Wooly Adelgid 
##                                 16                                 16 
##                               Mite                        Onion Thrip 
##                                 16                                 16 
##              Western Flower Thrips                       Corn Earworm 
##                                 15                                 14 
##                  Green Peach Aphid                          House Fly 
##                                 14                                 14 
##                          Ox Beetle                 Red Scale Parasite 
##                                 14                                 14 
##                 Spined Soldier Bug              Armoured Scale Family 
##                                 14                                 13 
##                   Diamondback Moth                      Eulophid Wasp 
##                                 13                                 13 
##                  Monarch Butterfly                      Predatory Bug 
##                                 13                                 13 
##              Yellow Fever Mosquito                Braconid Parasitoid 
##                                 13                                 12 
##                       Common Thrip       Eastern Subterranean Termite 
##                                 12                                 12 
##                             Jassid                         Mite Order 
##                                 12                                 12 
##                          Pea Aphid                   Pond Wolf Spider 
##                                 12                                 12 
##           Spotless Ladybird Beetle             Glasshouse Potato Wasp 
##                                 11                                 10 
##                           Lacewing            Southern House Mosquito 
##                                 10                                 10 
##            Two Spotted Lady Beetle                         Ant Family 
##                                 10                                  9 
##                       Apple Maggot                            (Other) 
##                                  9                                670
\end{verbatim}

\begin{quote}
Answer: The six most commonly studied species are Honey Bee, Parasitic
Wasp, Buff Trailed Bumblbee, Carniolan Honey Bee, Bumble Bee, and
Italian Honeybee. All of these species, except the parasitic wasp, is a
pollinator. Pollinators are important for plants to reproduce and grow
so studying the effects of neonicotinoids on these pollinators is of
interest to make sure the insecticide aren't harming their presence.
\end{quote}

\begin{enumerate}
\def\labelenumi{\arabic{enumi}.}
\setcounter{enumi}{7}
\tightlist
\item
  Concentrations are always a numeric value. What is the class of
  Conc.1..Author. in the dataset, and why is it not numeric?
\end{enumerate}

\begin{Shaded}
\begin{Highlighting}[]
\KeywordTok{class}\NormalTok{(Neonics}\OperatorTok{$}\NormalTok{Conc.}\DecValTok{1}\NormalTok{..Author.)}
\end{Highlighting}
\end{Shaded}

\begin{verbatim}
## [1] "factor"
\end{verbatim}

\begin{quote}
Answer: The class of Conc.1..Author is a factor because each of the
numbers have a / placed after it, thus, it can not be classed as
numeric.
\end{quote}

\hypertarget{explore-your-data-graphically-neonics}{%
\subsection{Explore your data graphically
(Neonics)}\label{explore-your-data-graphically-neonics}}

\begin{enumerate}
\def\labelenumi{\arabic{enumi}.}
\setcounter{enumi}{8}
\tightlist
\item
  Using \texttt{geom\_freqpoly}, generate a plot of the number of
  studies conducted by publication year.
\end{enumerate}

\begin{Shaded}
\begin{Highlighting}[]
\KeywordTok{ggplot}\NormalTok{(Neonics) }\OperatorTok{+}
\StringTok{  }\KeywordTok{geom_freqpoly}\NormalTok{(}\KeywordTok{aes}\NormalTok{(}\DataTypeTok{x=}\NormalTok{Publication.Year), }\DataTypeTok{color=}\StringTok{"Purple"}\NormalTok{)}\OperatorTok{+}
\StringTok{  }\KeywordTok{labs}\NormalTok{(}\DataTypeTok{x=}\StringTok{"Publication Year"}\NormalTok{, }\DataTypeTok{y=}\StringTok{"Count"}\NormalTok{)}
\end{Highlighting}
\end{Shaded}

\begin{verbatim}
## `stat_bin()` using `bins = 30`. Pick better value with `binwidth`.
\end{verbatim}

\includegraphics{A03_DataExploration_files/figure-latex/unnamed-chunk-6-1.pdf}

\begin{enumerate}
\def\labelenumi{\arabic{enumi}.}
\setcounter{enumi}{9}
\tightlist
\item
  Reproduce the same graph but now add a color aesthetic so that
  different Test.Location are displayed as different colors.
\end{enumerate}

\begin{Shaded}
\begin{Highlighting}[]
\KeywordTok{ggplot}\NormalTok{(Neonics) }\OperatorTok{+}
\StringTok{  }\KeywordTok{geom_freqpoly}\NormalTok{(}\KeywordTok{aes}\NormalTok{(}\DataTypeTok{x =}\NormalTok{ Publication.Year, }\DataTypeTok{color =}\NormalTok{ Test.Location ))}\OperatorTok{+}
\StringTok{  }\KeywordTok{labs}\NormalTok{(}\DataTypeTok{x=}\StringTok{"Publication Year"}\NormalTok{, }\DataTypeTok{y=}\StringTok{"Count"}\NormalTok{)}
\end{Highlighting}
\end{Shaded}

\begin{verbatim}
## `stat_bin()` using `bins = 30`. Pick better value with `binwidth`.
\end{verbatim}

\includegraphics{A03_DataExploration_files/figure-latex/unnamed-chunk-7-1.pdf}

Interpret this graph. What are the most common test locations, and do
they differ over time?

\begin{quote}
Answer: The most common test location was a lab, and the second most
common was in the field naturally. Between the 1990s and 2000s, field
natural was used more than the lab, but as time continued into the
2010s, lab as the test location increased greatly.
\end{quote}

\begin{enumerate}
\def\labelenumi{\arabic{enumi}.}
\setcounter{enumi}{10}
\tightlist
\item
  Create a bar graph of Endpoint counts. What are the two most common
  end points, and how are they defined? Consult the ECOTOX\_CodeAppendix
  for more information.
\end{enumerate}

\begin{Shaded}
\begin{Highlighting}[]
\KeywordTok{ggplot}\NormalTok{(Neonics)}\OperatorTok{+}
\StringTok{  }\KeywordTok{geom_bar}\NormalTok{(}\KeywordTok{aes}\NormalTok{(}\DataTypeTok{x =}\NormalTok{ Endpoint), }\DataTypeTok{fill =} \StringTok{"blue1"}\NormalTok{)}
\end{Highlighting}
\end{Shaded}

\includegraphics{A03_DataExploration_files/figure-latex/unnamed-chunk-8-1.pdf}

\begin{quote}
Answer: The two most common end points are NOEL and LOEL. NOEL stands
for no-observable-effect-level. This means that the higest dose
producing effects are not siginificantly different from the reponses of
controls. LOEL stands for lowest-observable-effect level. This means the
lowest dose producing effects that were significantly different from
reponse of controls.
\end{quote}

\hypertarget{explore-your-data-litter}{%
\subsection{Explore your data (Litter)}\label{explore-your-data-litter}}

\begin{enumerate}
\def\labelenumi{\arabic{enumi}.}
\setcounter{enumi}{11}
\tightlist
\item
  Determine the class of collectDate. Is it a date? If not, change to a
  date and confirm the new class of the variable. Using the
  \texttt{unique} function, determine which dates litter was sampled in
  August 2018.
\end{enumerate}

\begin{Shaded}
\begin{Highlighting}[]
\KeywordTok{class}\NormalTok{(Litter}\OperatorTok{$}\NormalTok{collectDate) }\CommentTok{#class is factor }
\end{Highlighting}
\end{Shaded}

\begin{verbatim}
## [1] "factor"
\end{verbatim}

\begin{Shaded}
\begin{Highlighting}[]
\CommentTok{#change from factor to date format}
\NormalTok{Litter}\OperatorTok{$}\NormalTok{collectDate <-}\StringTok{ }\KeywordTok{as.Date}\NormalTok{(Litter}\OperatorTok{$}\NormalTok{collectDate, }\DataTypeTok{format =} \StringTok{"%Y-%m-%d"}\NormalTok{)}
\KeywordTok{class}\NormalTok{(Litter}\OperatorTok{$}\NormalTok{collectDate) }\CommentTok{#class is a date}
\end{Highlighting}
\end{Shaded}

\begin{verbatim}
## [1] "Date"
\end{verbatim}

\begin{Shaded}
\begin{Highlighting}[]
\CommentTok{#Use Unique function to see what dates were sampled in August 2018}
\NormalTok{august_}\DecValTok{2018}\NormalTok{ <-}\StringTok{ }\KeywordTok{unique}\NormalTok{(}\KeywordTok{format}\NormalTok{(Litter}\OperatorTok{$}\NormalTok{collectDate, }\StringTok{"2018-08-%d"}\NormalTok{))}
\NormalTok{august_}\DecValTok{2018}
\end{Highlighting}
\end{Shaded}

\begin{verbatim}
## [1] "2018-08-02" "2018-08-30"
\end{verbatim}

\begin{enumerate}
\def\labelenumi{\arabic{enumi}.}
\setcounter{enumi}{12}
\tightlist
\item
  Using the \texttt{unique} function, determine how many plots were
  sampled at Niwot Ridge. How is the information obtained from
  \texttt{unique} different from that obtained from \texttt{summary}?
\end{enumerate}

\begin{Shaded}
\begin{Highlighting}[]
\KeywordTok{unique}\NormalTok{(Litter}\OperatorTok{$}\NormalTok{plotID)}
\end{Highlighting}
\end{Shaded}

\begin{verbatim}
##  [1] NIWO_061 NIWO_064 NIWO_067 NIWO_040 NIWO_041 NIWO_063 NIWO_047
##  [8] NIWO_051 NIWO_058 NIWO_046 NIWO_062 NIWO_057
## 12 Levels: NIWO_040 NIWO_041 NIWO_046 NIWO_047 NIWO_051 ... NIWO_067
\end{verbatim}

\begin{quote}
Answer: There were 12 plots that were sampled at Niwot Ridge.
Information from Unique is different from Summary because Unique returns
the names of all the different entries in the column that you want.
Whereas Summary returns the number of times each of those entries comes
up in the column.
\end{quote}

\begin{enumerate}
\def\labelenumi{\arabic{enumi}.}
\setcounter{enumi}{13}
\tightlist
\item
  Create a bar graph of functionalGroup counts. This shows you what type
  of litter is collected at the Niwot Ridge sites. Notice that litter
  types are fairly equally distributed across the Niwot Ridge sites.
\end{enumerate}

\begin{Shaded}
\begin{Highlighting}[]
\KeywordTok{ggplot}\NormalTok{(Litter)}\OperatorTok{+}
\StringTok{  }\KeywordTok{geom_bar}\NormalTok{(}\KeywordTok{aes}\NormalTok{(}\DataTypeTok{x =}\NormalTok{ functionalGroup), }\DataTypeTok{fill =} \StringTok{"firebrick4"}\NormalTok{)}
\end{Highlighting}
\end{Shaded}

\includegraphics{A03_DataExploration_files/figure-latex/unnamed-chunk-11-1.pdf}

\begin{enumerate}
\def\labelenumi{\arabic{enumi}.}
\setcounter{enumi}{14}
\tightlist
\item
  Using \texttt{geom\_boxplot} and \texttt{geom\_violin}, create a
  boxplot and a violin plot of dryMass by functionalGroup.
\end{enumerate}

\begin{Shaded}
\begin{Highlighting}[]
\CommentTok{#boxplot }
\KeywordTok{ggplot}\NormalTok{(Litter) }\OperatorTok{+}
\StringTok{  }\KeywordTok{geom_boxplot}\NormalTok{(}\KeywordTok{aes}\NormalTok{(}\DataTypeTok{x =}\NormalTok{ functionalGroup, }\DataTypeTok{y =}\NormalTok{ dryMass))}
\end{Highlighting}
\end{Shaded}

\includegraphics{A03_DataExploration_files/figure-latex/unnamed-chunk-12-1.pdf}

\begin{Shaded}
\begin{Highlighting}[]
\CommentTok{#violin }
\KeywordTok{ggplot}\NormalTok{(Litter)}\OperatorTok{+}
\StringTok{  }\KeywordTok{geom_violin}\NormalTok{(}\KeywordTok{aes}\NormalTok{(}\DataTypeTok{x =}\NormalTok{ functionalGroup, }\DataTypeTok{y =}\NormalTok{ dryMass))}
\end{Highlighting}
\end{Shaded}

\includegraphics{A03_DataExploration_files/figure-latex/unnamed-chunk-12-2.pdf}

Why is the boxplot a more effective visualization option than the violin
plot in this case?

\begin{quote}
Answer: The violin plot does not show the distribution of the dry mass
of each functional group as well as the boxplot does. The boxplot shows
outliers and the skew of the data more so than the violin plot.
\end{quote}

What type(s) of litter tend to have the highest biomass at these sites?

\begin{quote}
Answer: Needles have a high biomass at these sites followed by litter
that is mixed.
\end{quote}


\end{document}
